 \documentclass[10pt,a4paper]{scrartcl}

\usepackage{graphicx}
\usepackage{amsmath}
\usepackage{amsfonts}
\usepackage{amssymb}
\usepackage{bm}
\let\mathbf\bm
\usepackage[left=2.5cm, right=3cm, top=2cm]{geometry}

\usepackage[T1]{fontenc}
\usepackage[utf8]{inputenc}
\usepackage[magyar]{babel}
\usepackage{lmodern}

\usepackage{placeins}
\usepackage{subcaption}
\usepackage{epstopdf}
\usepackage{media9}
\usepackage{xcolor}
\usepackage[hidelinks,unicode]{hyperref}
\hypersetup{
    colorlinks,
    linkcolor={red!50!black},
    citecolor={blue!50!black},
    urlcolor={blue!80!black}
}
\usepackage[most]{tcolorbox}
\tcbset{highlight math style={
	enhanced,
	boxrule=0pt,
	colframe=blue,
	colback=white,
	arc=0pt,
	boxrule=0pt,
	left={3pt},
	right={3pt},
	top={3pt},
	bottom={3pt}}}

\usepackage{cleveref}
\newcommand*\Laplace{\mathop{}\!\mathbin\bigtriangleup}

\begin{document}
\tableofcontents

\section{Ismertető}
Szeretnénk egy olyan funkcionált felírni, aminek a megfelelő funkcionál-deriváltjaiból kijönnek a megfelelő mozgásegyenletek. A funkcionál függvénye az alábbiaknak:
\[P=P\left[ {\chi ,{\rho _ + },{\rho _ - },c} \right].\]
Úgy képezzük az eredő potenciált, hogy összeadjuk a rugalmas közeg - diszlokációk önkonzisztens potenciálját, a diszlokációk mintázatképződésért felelős potenciált, illetve az oldott atomok önkölcsönhatásból és a nyomással való kölcsönhatásból származtatott potenciálokat.
\[P = {P_{sc}} + {P_{corr}} + {P_{c - cp}}\]
A $P_{c - cp}$ tag lehet a $c\ln \left( c \right)$ alakú $M_c \sim c$ mobilitással, vagy ${\left( {c - {c_\infty }} \right)^2}$ alakú konstans $M_c$ mobilitással. Core regularizációs tagokat megpróbálhatjuk 0-nak venni, $a' = a = 0$.
\begin{multline} \label{eq:plastic_pot}
P = \int \left[ \underbrace { - \frac{{1 - \nu }}{{4\mu }}{{\left( {\Laplace \chi } \right)}^2} + b\chi \frac{\partial \kappa}{\partial y} + \overbrace{a^2\left| {\nabla \left( \Laplace \chi  \right)} \right|^2}^{\text{core reg.}}}_{{\text{önkonzisztens tér}}} + \underbrace {G{b^2}A\rho \ln \left( {\frac{\rho }{{{\rho _0}}}} \right) + \frac{{G{b^2}D}}{2}\frac{{{\kappa ^2}}}{\rho }}_{{\text{diszlokációk korrelációjához}}} + \right. \\ 
\left. + \underbrace {\frac{\alpha_c}{2} {{\left( {c - {c_\infty }} \right)}^2}}_{\substack{\text{oldott atom}\\\text{önkh.}}} - \underbrace {\beta c\left( {\Laplace \chi } \right)}_{\substack{\text{oldott atom}\\\text{nyomás kh-a}}} + \overbrace {a{'^2}\nabla \left( \Laplace \chi  \right) \cdot \nabla c}^{{\text{core reg.}}}\right] {d^2}r.
\end{multline}
Az egyenlet $\left[ {\chi ,{\rho _ + },{\rho _ - },c} \right]$-nak a függvénye, és az egyes argumentumok szerinti parciális deriváltakból 1-1 egyenletet kapunk. Sorrendben ezek az alábbiak.
\subsection{Másodrendű feszültségtenzor}
A $\chi$ , $\kappa$ és immár $c$ közötti kapcsolatot kifejező egyenlet: 
\begin{equation}
\frac{{\delta P}}{{\delta \chi }} = 0.
\end{equation}
Vagyis feltesszük, hogy a rendszer azonnal beáll abba az állapotba, ahol az elasztikus energiája minimális. Ez jó közelítés, amíg a karakterisztikus sebességek kisebbek a hangsebességnél. Ennek segítségével a többi egyenletből (\cref{eq:rho_eom,eq:kappa_eom,eq:c_conti}) eliminálható $\chi$.
\subsection{Diszlokációsűrűségek} A $\kappa$ és $\rho$ időfejlődésére vonatkozó egyenletek:
\begin{align} \label{eq:rho_eom}
  {\partial _t}\rho  & =  - {\partial _x}\left\{ {\kappa {M_0} \cdot \left( { - {\partial _x}\frac{{\delta P}}{{\delta \kappa }}} \right) + \rho {M_0} \cdot \left( { - {\partial _x}\frac{{\delta P}}{{\delta \rho }}} \right)} \right\} \\
  {\partial _t}\kappa  & =  - {\partial _x}\left\{ {\rho M\left( { - {\partial _x}\frac{{\delta P}}{{\delta \kappa }}} \right)\quad  + \kappa {M_0} \cdot \left( { - {\partial _x}\frac{{\delta P}}{{\delta \rho }}} \right)} \right\}. \label{eq:kappa_eom}
\end{align}
A korábbi cikkekben és a Mesoscale könyveben az előjelekkel egyszerűsítve van, viszont abban az esetben a mobilitás hibásan szerepel. A helyes mobilitást viszont nem lehet felírni intuitív módon pozitív argumentumra. \Cref{eq:rho_eom,eq:kappa_eom}-ben a $\partial_x$ úgy hat a $\kappa /\rho $-ra, hogy csak $\kappa$-ban derivál, mert a magasabb rendű tagokat elhagyjuk. Kicsi $\kappa /\rho $, ${\partial _x}\rho /{\rho ^{3/2}}$ és $\frac{{c - {c_\infty }}}{{{c_\infty }}}$, azaz ezek szorzatait és magasabb rendű hatványait elhagyjuk.

A mobilitás egy nemtriviális függvénye az argumentumának, de mi csak ott vizsgálódunk, ahol 
\begin{equation} \label{eq:nontriv_mobility}
M\left( { - {\partial _x}\frac{{\delta P}}{{\delta \kappa }}} \right) = {M_0}\left[ { - {\partial _x}\frac{{\delta P}}{{\delta \kappa }} - \alpha \mu {b^2}\sqrt \rho} \right].
\end{equation}
Ebben már elhanyagoltunk egy $ \kappa ^2 / \rho ^2 $-es tagot.

\subsection{Oldott atom koncentrációja} Van egy egyenletünk $c$ megmaradására:
\begin{equation} \label{eq:c_conti}
\partial_t c =  \nabla \left( {{M_c} \cdot \nabla \frac{{\delta P}}{{\delta c}}} \right).
\end{equation}
Illetve, ha gyors diffúziót feltételezünk, akkor $c$-ben az egyensúly rövid idő alatt úgy áll be, hogy
\[\frac{{\delta P}}{{\delta c}} = \mu_c \] teljesüljön.

\subsection{Megoldási terv}
Az \cref{eq:rho_eom,eq:kappa_eom,eq:c_conti}-en lineáris stabilitáselemzést lehet végezni, ami egy $3\times3$-as mátrix vizsgálatához vezet: a mátrix determinánsa legyen 0.



\section{Oldott atom hatása core regularizáció nélkül}
\subsection{A mozgásegyeneltek felírása}
Nézzük meg, hogy core regularizáció nélkül mire jutunk. A diszlokációk korrelációjából származó tagokat is tartalmaz ez a potenciál, így reméljük, hogy hátha megfelelően elkeni a core szinguláris problémáját. A potenciált tehát $a=0$ és $a'=0$ mellett:
\begin{equation} \label{eq:plastic_pot_wo_core_reg}
P_s = \int { \underbrace{- \frac{{1 - \nu }}{{4\mu }}{{\left( {\Laplace\chi } \right)}^2}}_{P_{s,1}} + \underbrace{b\chi \frac{{\partial \kappa }}{{\partial y}}}_{P_{s,2}} + \underbrace{G{b^2}A\rho \ln \left( {\frac{\rho }{{{\rho _0}}}} \right)}_{P_{s,3}} + \underbrace{\frac{{G{b^2}D}}{2}\frac{{{\kappa ^2}}}{\rho }}_{P_{s,4}} + \underbrace{\frac{\alpha_c}{2} {\left( {c - {c_\infty }} \right)^2}}_{P_{s,5}} + \underbrace{\beta c\left(- {\Laplace\chi } \right)}_{P_{s,6}}}{d^2}r.
\end{equation}
\subsubsection{Másodrendű feszültségtenzor}
A $\frac{{\delta {P_s}}}{{\delta \chi }} = 0$ feltételből:
\begin{equation} \label{eq:chi}
\tcbhighmath[borderline={1pt}{0pt}{blue,solid}]{- \frac{{1 - \nu }}{{2\mu }}{\Delta ^2}\chi  + b\frac{{\partial \kappa }}{{\partial y}} - \beta  \cdot \Delta c = 0.}
\end{equation}
\subsubsection{Diszlokáció sűrűségek} $\kappa$ kontinuitási egyenletéhez \cref{eq:kappa_eom} egyenletbe helyettesítsünk be az $M$ mobilitásával a \cref{eq:nontriv_mobility}-val, $\rho$-é változatlan:
\begin{gather} \label{eq:rho_kappa_conti}
\tcbhighmath[borderline={1pt}{0pt}{blue,dashed}]{
\begin{aligned}{\partial _t}\rho  &  = {M_0}{\partial _x}\left\{ {\kappa \cdot \left( {{\partial _x}\frac{{\delta P}}{{\delta \kappa }}} \right) + \rho \cdot {\partial _x}\frac{{\delta P}}{{\delta \rho }}} \right\}, \\ 
  {\partial _t}\kappa  &  = {M_0}{\partial _x}\left\{ {\rho \left[ {{\partial _x}\frac{{\delta P}}{{\delta \kappa }} + \alpha \mu {b^2}\sqrt \rho } \right] + \kappa  \cdot {\partial _x}\frac{{\delta P}}{{\delta \rho }}} \right\}.
\end{aligned}}
\end{gather}
Ennek megadásához el kell végezni a $\frac{{\delta P}}{{\delta \rho }}$ és $\frac{{\delta P}}{{\delta \kappa }}$ deriválást.\paragraph{Először $\kappa$-ra} az egyes tagokon elvégezve:
\begin{align*}
  {P_{s,2}}\left[ {\kappa  + \delta \kappa } \right] &  = \int_{{L^2}} {b\chi \left( {\frac{\partial }{{\partial y}}\left( {\kappa  + \delta \kappa } \right)} \right)dA}  \\ 
   &  = \underbrace {\int_{{L^2}} {b\chi \left( {\frac{\partial }{{\partial y}}\kappa } \right)dA} }_{{P_{s,2}}\left[ \kappa  \right]} + \int_{{L^2}} {b\chi \left( {\frac{\partial }{{\partial y}}\delta \kappa } \right)dA}  \\ 
   &  = {P_{s,2}}\left[ \kappa  \right] + \int_{{L^2}} {\int\limits_{ - \infty }^\infty  {b\chi \left( {\frac{\partial }{{\partial y}}\delta \kappa } \right)} dy} dx \\ 
   &  = {P_{s,2}}\left[ \kappa  \right] + \int_{{L^2}} {\underbrace {\int\limits_{ - \infty }^\infty  {b\left( {\frac{\partial }{{\partial y}}\left( {\chi  \cdot \delta \kappa } \right)} \right)} dy}_{\left. {\chi  \cdot \delta \kappa } \right|_{ - \infty }^\infty  = 0}} dx - \int_{{L^2}} {\int\limits_{ - \infty }^\infty  {b\left( {\frac{\partial }{{\partial y}}\left( \chi  \right) \cdot \delta \kappa } \right)dy} dx}  \\ 
   &  = {P_{s,2}}\left[ \kappa  \right] - \int_{{L^2}} {b\left( {\frac{\partial }{{\partial y}}\left( \chi  \right) \cdot \delta \kappa } \right)dA}, \text{ tehát}
\end{align*}
\begin{equation} \label{eq:ps2_kappa}
{P_{s,2}}\left[ {\kappa  + \delta \kappa } \right] - {P_{s,2}}\left[ \kappa  \right] =  - \int_{{L^2}} {b\left( {\frac{\partial }{{\partial y}}\left( \chi  \right) \cdot \delta \kappa } \right)dA}  \Rightarrow \tcbhighmath[borderline={0.5pt}{0pt}{blue,dotted}]{ \frac{{\delta {P_{s,2}}\left[ \kappa  \right]}}{{\delta \kappa }} =  - b\frac{{\partial \chi }}{{\partial y}}}
\end{equation}
\begin{equation} \label{eq:ps4_kappa}
{P_{s,4}}\left[ \kappa  \right] = \int_{{L^2}} {\frac{{G{b^2}D}}{2}\frac{{{\kappa ^2}}}{\rho }dA}  \Rightarrow \tcbhighmath[borderline={0.5pt}{0pt}{blue,dotted}]{\frac{{\delta {P_{s,4}}\left[ \kappa  \right]}}{{\delta \kappa }} = G{b^2}D\frac{\kappa }{\rho }}
\end{equation}
A \cref{eq:ps2_kappa,eq:ps4_kappa} egyenletekből
\begin{equation} \label{eq:ps_kappa}
\tcbhighmath[borderline={1pt}{0pt}{blue,dashed}]{\frac{{\delta {P_s}\left[ \kappa  \right]}}{{\delta \kappa }} = G{b^2}D\frac{\kappa }{\rho } - b\frac{{\partial \chi }}{{\partial y}}.}
\end{equation}
\paragraph{Elvégezve $\rho$-ra} az egyes tagokon:
\[\begin{aligned}
  {P_{s,3}}\left[ {\rho  + \delta \rho } \right] &  = \int_{{L^2}} {G{b^2}A \cdot \left( {\rho  + \delta \rho } \right)\underbrace {\ln \left( {\left( {\rho  + \delta \rho } \right)/{\rho _0}} \right)}_{\ln \left( {\rho /{\rho _0}} \right) + \delta \rho /\rho }} dA \\ 
   &  = {P_{s,3}}\left[ \rho  \right] + \int_{{L^2}} {G{b^2}A \cdot \delta \rho  \cdot \ln \left( {\rho /{\rho _0}} \right)} dA + \int_{{L^2}} {G{b^2}A \cdot \delta \rho } dA, \\ 
\end{aligned} \]
amelyből egyrészt
\begin{equation} \label{eq:ps3_rho}
\tcbhighmath[borderline={0.5pt}{0pt}{blue,dotted}]{\frac{{{\delta P_{s,3}}\left[ \rho  \right]}}{{\delta \rho }} = G{b^2}A \cdot \ln \left( {\rho /{\rho _0}} \right) + G{b^2}A,}
\end{equation}
másrészt
\[\begin{aligned}
  {P_{s,4}}\left[ {\rho  + \delta \rho } \right] &  = \int_{{L^2}} {\frac{{G{b^2}D}}{2} \cdot \underbrace {\frac{{{\kappa ^2}}}{{\rho  + \delta \rho }}}_{\frac{{{\kappa ^2}}}{\rho } - \frac{{{\kappa ^2}}}{{{\rho ^2}}}\delta \rho \approx \frac{\kappa^2}{\rho }}} dA \\ 
   &  = {P_{s,4}}\left[ \rho  \right], \\ 
\end{aligned} \]
amelyből
\begin{equation} \label{eq:ps4_rho}
\tcbhighmath[borderline={0.5pt}{0pt}{blue,dotted}]{\frac{{{\delta P_{s,4}}\left[ \rho  \right]}}{{\delta \rho }} =  0,}
\end{equation}
így \cref{eq:ps3_rho,eq:ps4_rho} egyenleteket felhasználva kapjuk:
\begin{equation} \label{eq:ps_rho}
\tcbhighmath[borderline={1pt}{0pt}{blue,dashed}]{\frac{{{\delta P_s}\left[ \rho  \right]}}{{\delta \rho }} = G{b^2}A \cdot \ln \left( {\rho /{\rho _0}} \right) + G{b^2}A .}
\end{equation}
Be kell helyettesíteni \cref{eq:rho_kappa_conti}-ba \cref{eq:ps_kappa,eq:ps_rho}-vel. Ha ${{\partial _x}\frac{{\delta P}}{{\delta \kappa }}} > 0$, akkor:
\[\begin{aligned}
  {\partial _t}\rho  &  = {M_0}{\partial _x}\left\{ {\kappa  \cdot {\partial _x}\left( {G{b^2}D \cdot \underbrace {\frac{\kappa }{\rho }}_{{\partial _x}\frac{\kappa }{\rho } \approx \frac{1}{\rho }{\partial _x}\kappa } - b\frac{{\partial \chi }}{{\partial y}}} \right) + \rho {\partial _x}\left( {G{b^2}A \cdot \ln \left( {\frac{\rho }{{{\rho _0}}}} \right) + G{b^2}A} \right)} \right\} \\ 
   &  = {M_0}b{\partial _x}\left\{ {\kappa  \cdot \left( {GbD\frac{{{\partial _x}\kappa }}{\rho } - \underbrace {\frac{{{\partial ^2}\chi }}{{\partial x\partial y}}}_{{\tau _{{\text{mf}}}}}} \right) + \rho  \cdot GbA\frac{{{\partial _x}\rho }}{\rho }} \right\}, \\ 
  {\partial _t}\kappa  &  = {M_0}{\partial _x}\left\{ {\rho \left[ {{\partial _x}\left( {G{b^2}D\frac{\kappa }{\rho } - b\frac{{\partial \chi }}{{\partial y}}} \right) + \alpha \mu {b^2}\sqrt \rho } \right] + \kappa {\partial _x}\left( {G{b^2}A\ln \left( {\frac{\rho }{{{\rho _0}}}} \right) + G{b^2}A} \right)} \right\} \\ 
   &  = {M_0}b{\partial _x}\left\{ {\rho \left( {GbD\frac{{{\partial _x}\kappa }}{\rho } - \underbrace {\frac{{{\partial ^2}\chi }}{{\partial x\partial y}}}_{{\tau _{{\text{mf}}}}} + \alpha \mu {b^2}\sqrt \rho } \right) + \kappa  \cdot GbA\frac{{{\partial _x}\rho }}{\rho }} \right\}, \\ 
\end{aligned} \]
vagyis összesen:
\begin{gather} \label{eq:rho_kappa}
\tcbhighmath[borderline={1pt}{0pt}{blue,solid}]{
\begin{aligned}
  {\partial _t}\rho  &  =  - {M_0}b{\partial _x}\left\{ {\kappa {\tau _{{\text{mf}}}} - \frac{\kappa }{\rho }GbD{\partial _x}\kappa  - GbA{\partial _x}\rho } \right\}, \\ 
  {\partial _t}\kappa  &  =  - {M_0}b{\partial _x}\left\{ {\rho {\tau _{{\text{mf}}}} - \rho \alpha \mu b\sqrt \rho - GbD{\partial _x}\kappa  - \frac{\kappa }{\rho } GbA{\partial _x}\rho } \right\}.
  \end{aligned}}
\end{gather}
\subsubsection{Oldott atom koncentráció} A koncentrációra vonatkozó egyenlethez $\frac{{\delta {P_s}\left( c \right)}}{{\delta c}}$-t kell kiszámolni.
\begin{align}
  {P_{s,5}}\left[ {c + \delta c} \right] &  = \int_{{L^2}} {\frac{\alpha_c}{2}  \cdot \underbrace {{{\left( {c + \delta c - {c_\infty }} \right)}^2}}_{{{\left( {c - {c_\infty }} \right)}^2} + 2\left( {c - {c_\infty }} \right) \cdot \delta c + {{\left( {\delta c} \right)}^2}}dA} \nonumber \\ 
   &  = {P_{s,5}}\left[ c \right] + \int_{{L^2}} {\frac{\alpha_c}{2}  \cdot 2\left( {c - {c_\infty }} \right)\delta cdA}  \Rightarrow\tcbhighmath[borderline={0.5pt}{0pt}{blue,dotted}]{ \frac{{\delta {P_{s,5}}\left[ c \right]}}{{\delta c}} = \alpha_c  \cdot \left( {c - {c_\infty }} \right)}
\end{align}
\begin{equation}
{P_{s,6}}\left[ {c + \delta c} \right] = \int_{{L^2}} {\beta \left( {c + \delta c} \right)\left( { - \Delta \chi } \right)} dA \Rightarrow \tcbhighmath[borderline={0.5pt}{0pt}{blue,dotted}]{\frac{{\delta {P_{s,6}}\left[ c \right]}}{{\delta c}} =  - \beta  \cdot \Delta \chi}
\end{equation}
\begin{equation} \label{eq:c_pot}
\tcbhighmath[borderline={1pt}{0pt}{blue,dashed}]{\frac{{\delta {P_s}\left[ c \right]}}{{\delta c}} = \alpha_c  \cdot \left( {c - {c_\infty }} \right) - \beta  \cdot \Delta \chi }
\end{equation}
A koncentráció \told\ref{eq:c_conti}+as{} kontinuitási egyenletéből megkapjuk az időfejődést:
\[\partial_t c =  - \nabla \left( {{M_c} \cdot \nabla \left( {{\alpha _c} \cdot \left( {c - {c_\infty }} \right) - \beta  \cdot \Delta \chi } \right)} \right),\]
amelyben $M_c$ és $c_{\infty}$ állandó térben, így
\begin{equation} \label{eq:c}
\tcbhighmath[borderline={1pt}{0pt}{blue,solid}]{
\partial_t c =  {M_c}{\alpha _c}\Delta c - \beta {M_c}\Delta^2 \chi.}
\end{equation}

\subsection{Mozgásegyenletek megoldása lineáris stabilitáselemzéssel}
\Cref{eq:chi,eq:rho_kappa,eq:c} egyenletek együtt, $\partial_x \partial_y \chi = {\tau _{{\text{mf}}}}$ jelöléssel, $\kappa^2/\rho^2$-es tag elhanyagolásával:
\begin{gather} \label{eq:eom}
\tcbhighmath[borderline={1pt}{0pt}{black,solid}]{
\begin{aligned}
0 & = - \frac{{1 - \nu }}{{2\mu }}{\Delta ^2}\chi  + b\cdot\partial_y\kappa - \beta  \cdot \Delta c \\ 
    {\partial _t}\rho  &  =  - {M_0}b{\partial _x}\left\{ {\kappa {\tau _{{\text{mf}}}} - \frac{\kappa }{\rho }GbD{\partial _x}\kappa  - GbA{\partial _x}\rho } \right\} \\ 
  {\partial _t}\kappa  &  =  - {M_0}b{\partial _x}\left\{ {\rho {\tau _{{\text{mf}}}} - \rho \alpha \mu b\sqrt \rho - GbD{\partial _x}\kappa  - \frac{\kappa }{\rho } GbA{\partial _x}\rho } \right\} \\
  \partial_t c & =  {M_c}{\alpha _c}\Delta c - \beta {M_c}\Delta^2 \chi.
\end{aligned}  }
\end{gather}
Ezek megoldását keressük egy egyensúlyi állapot körül,
\[\begin{aligned}
  \chi \left( {t,{\mathbf{r}}} \right) &  = {\chi _0} + \delta \chi \left( {t,{\mathbf{r}}} \right) \\ 
  \rho \left( {t,{\mathbf{t}}} \right) &  = {\rho _0} + \delta \rho \left( {t,{\mathbf{r}}} \right) \\ 
  \kappa \left( {t,{\mathbf{r}}} \right) &  = {\kappa _0} + \delta \kappa \left( {t,{\mathbf{r}}} \right) \\ 
  c\left( {t,{\mathbf{r}}} \right) &  = {c_0} + \delta c\left( {t,{\mathbf{r}}} \right).
\end{aligned} \]
Ha konstansok volnának, azok kielégítenék az egyenleteket, és a korábbi jelölések konvenciója szerint 
\[\begin{aligned}
  {\partial _x}{\partial _y}\chi  = {\sigma _{xy}} = {\tau _0} \Rightarrow {\chi _0} &  = {\tau _0} \cdot xy \\ 
  \left\langle \rho  \right\rangle  &  = {\rho _0} \\ 
  \left\langle \kappa  \right\rangle  = 0 \Rightarrow {\kappa _0} &  = 0 \\ 
  \left\langle c \right\rangle  = {c_\infty } \Rightarrow {c_0} &  = {c_\infty }. \\ 
\end{aligned} \]
Az egyes variációk idő és térfüggését $\exp \left( {\frac{\lambda }{{{t_0}}} \cdot t + i\sqrt {{\rho _0}}  \cdot {\mathbf{kr}}} \right)$ alakban keresve a vizsgált alakok:
\begin{gather} \label{eq:perturbative_solution}
\tcbhighmath[borderline={1pt}{0pt}{black,solid}]{
\left( {\begin{array}{*{20}{c}}
  \chi  \\ 
  \rho  \\ 
  \kappa  \\ 
  c 
\end{array}} \right)\left( {t,{\mathbf{r}}} \right) = \left( {\begin{array}{*{20}{c}}
  {{\tau _0} \cdot xy} \\ 
  {{\rho _0}} \\ 
  0 \\ 
  {{c_\infty }} 
\end{array}} \right) + \left( {\begin{array}{*{20}{c}}
  {\delta {\chi _0}} \\ 
  {\delta {\rho _0}} \\ 
  {\delta {\kappa _0}} \\ 
  {\delta {c_0}} 
\end{array}} \right) \cdot \exp \left( {\frac{\lambda }{{{t_0}}} \cdot t + i\sqrt {{\rho _0}}  \cdot {\mathbf{kr}}} \right),}
\end{gather}
amelyben
\begin{equation} \label{eq:t_0_def}
1/{t_0} = {b^2}G{\rho _0}{M_0}.
\end{equation}
A behelyettesítés után a deriváló operátorok az alábbi sajátértéket veszik fel:
\begin{align} \label{eq:sajat_ertek_op}
  {\partial _t} &  \to \lambda /t_0 \\ 
  {\partial _x} &  \to i\sqrt {{\rho _0}}  \cdot {k_x} \\ 
  {\partial _y} &  \to i\sqrt {{\rho _0}}  \cdot {k_y} \\ 
  \Delta  &  \to {\left( {i\sqrt {{\rho _0}} } \right)^2} \cdot \left( {k_x^2 + k_y^2} \right) =  - {\rho _0} \cdot {k^2} \\ 
  {\Delta ^2} &  \to {\left( {i\sqrt {{\rho _0}} } \right)^4} \cdot \left( {k_x^4 + k_y^4 + 2k_x^2k_y^2} \right) = \rho _0^2 \cdot {k^4}. 
\end{align}

\subsubsection{Másodrendű feszültségtenzor}
\[\begin{aligned}
  0 &  =  - \frac{{1 - \nu }}{{2\mu }}{\Delta ^2}\left( {\delta \chi } \right) + b \cdot {\partial _y}\left( {\delta \kappa } \right) - \beta  \cdot \Delta \left( {\delta c} \right) \\ 
   &  =  - \frac{1}{{4\pi G}}\rho _0^2 \cdot {k^4}\left( {\delta \chi } \right) + b \cdot i\sqrt {{\rho _0}}  \cdot {k_y}\left( {\delta \kappa } \right) + \beta  \cdot {\rho _0} \cdot {k^2}\left( {\delta c} \right) \\ 
\end{aligned} \]
Ebből
\begin{equation} \label{eq:lin_ed_chi}
\tcbhighmath[borderline={1pt}{0pt}{blue,solid}]{
\delta \chi  = i\frac{{4\pi Gb}}{{\rho _0^{3/2}}}\frac{{{k_y}}}{{{k^4}}} \cdot \delta \kappa  + \frac{{4\pi G}}{{{\rho _0}}}\beta \frac{1}{{{k^2}}} \cdot \delta c.}
\end{equation}

\subsubsection{Diszlokációsűrűségek}
A teljes diszlokációsűrűség fejlődésére:
\[\begin{aligned}
  \frac{\lambda }{{{t_0}}}\delta \rho  &  =  - {M_0}b{\partial _x}\left\{ {\underbrace {\delta \kappa  \cdot {\partial _x}{\partial _y}\left( {xy \cdot {\tau _0} + \delta \chi } \right)}_{\delta \kappa  \cdot {\tau _0} + \delta \kappa  \cdot \delta \chi  \cdot  \ldots } - \underbrace {\frac{{\delta \kappa }}{\rho }GbD{\partial _x}\delta \kappa }_{ \propto \delta {\kappa ^2}} - GbA{\partial _x}\delta \rho } \right\} \\ 
   &  =  - {M_0}bi\sqrt {{\rho _0}}  \cdot {k_x}\left\{ {\delta \kappa  \cdot {\tau _0} - GbAi\sqrt {{\rho _0}}  \cdot {k_x}\delta \rho } \right\} \\ 
\end{aligned} \]
Ebből \aref{eq:t_0_def}.\ egyenlet szerint az idővel visszahelyettesítve:
\begin{equation} \label{eq:lin_ed_rho}
\tcbhighmath[borderline={1pt}{0pt}{blue,solid}]{
0 = \left( {\lambda  + Ak_x^2} \right) \cdot \delta \rho  + i\frac{{{\tau _0}}}{{bG\sqrt {{\rho _0}} }}{k_x} \cdot \delta \kappa }
\end{equation}
Az előjeles diszlokációsűrűség fejlődésére pedig:
\[\begin{aligned}
  \frac{\lambda }{{{t_0}}}\delta\kappa  &  =  - {M_0}b{\partial _x}\left\{ \begin{gathered}
  \underbrace {\left( {{\rho _0} + \delta \rho } \right) \cdot {\partial _x}{\partial _y}\left( {xy \cdot {\tau _0} + \delta \chi } \right)}_{\underbrace {{\rho _0}{\tau _0}}_{{\text{konst}}{\text{.}}} + \delta \rho  \cdot {\tau _0} + {\rho _0}{\partial _x}{\partial _y}\chi  + \underbrace {\delta \rho  \cdot \delta \chi  \cdot  \ldots }_{{\text{magasabb rend}}}} - \left( {{\rho _0} + \delta \rho } \right)\alpha \mu b\underbrace {\sqrt {{\rho _0} + \delta \rho } }_{ \approx \left[ {1 + \delta \rho /\left( {2{\rho _0}} \right)} \right]\sqrt {{\rho _0}} } -  \hfill \\
  \quad \quad \quad \quad \quad \quad \quad \quad \quad \quad \quad \quad  - GbD{\partial _x}\delta\kappa  - \underbrace {\frac{{\delta \kappa }}{{{\rho _0} + \delta \rho }}GbA{\partial _x}\left( {{\rho _0} + \delta \rho } \right)}_{{\text{konst}}{\text{. deriv - ja}} + {\text{magasabb rend}}} \hfill \\ 
\end{gathered}  \right\} \\ 
   & =  - {M_0}b{\partial _x}\left\{ {\delta \rho  \cdot {\tau _0} + {\rho _0}{\partial _x}{\partial _y}\delta \chi  - \underbrace {\left( {{\rho _0} + \delta \rho } \right)\alpha \mu b\left[ {1 + \delta \rho /\left( {2{\rho _0}} \right)} \right]\sqrt {{\rho _0}} }_{\frac{3}{2}\alpha \mu b \cdot \delta \rho } - GbD{\partial _x}\delta \kappa } \right\} \\ 
   &  =   - {M_0}bi\sqrt {{\rho _0}} {k_x}\left\{ {\left( {{\tau _0} - \frac{3}{2}\alpha \mu b\sqrt {{\rho _0}} } \right) \cdot \delta \rho  + {\rho _0}i\sqrt {{\rho _0}} {k_x}i\sqrt {{\rho _0}} {k_y}\delta \chi  - GbDi\sqrt {{\rho _0}} {k_x}\delta \kappa } \right\} \\ 
\end{aligned} \]
Ebből
\begin{equation} \label{eq:lin_ed_kappa}
\tcbhighmath[borderline={1pt}{0pt}{blue,solid}]{
0 = \lambda \delta \kappa  + i\left( {\frac{{{\tau _0}}}{{bG\sqrt {{\rho _0}} }} - \frac{3}{2}\frac{{\alpha \mu }}{G}} \right){k_x} \cdot \delta \rho  - i\frac{{\rho _0^{3/2}}}{{bG}}k_x^2{k_y} \cdot \delta \chi  + Dk_x^2 \cdot \delta \kappa .}
\end{equation}

\subsubsection{Oldott atom koncentráció}
\[\begin{aligned}
  \frac{\lambda }{{{t_0}}}\delta c &  =  {M_c}{\alpha _c}\Delta \left( {\delta c} \right) - \beta {M_c}{\Delta ^2}\left( {xy \cdot {\tau _0} + \delta \chi } \right) \\ 
   &  =  {M_c}{\alpha _c}\left( { - {\rho _0} \cdot {k^2}} \right)\delta c - \beta {M_c}\left( {\rho _0^2 \cdot {k^4}} \right) \cdot \delta \chi  \\ 
   &  = - {M_c}{\alpha _c}{\rho _0}{k^2} \cdot \delta c - \beta {M_c}\rho _0^2{k^4} \cdot \delta \chi  \\ 
\end{aligned} \]
Ebből
\begin{equation} \label{eq:lin_ed_c}
\tcbhighmath[borderline={1pt}{0pt}{blue,solid}]{
0 = \left( {\lambda  + \frac{{{M_c}{\alpha _c}{k^2}}}{{{M_0}{b^2}G}}} \right) \cdot \delta c + \frac{{\beta {M_c}{\rho _0}}}{{M_0 {b^2}G}}{k^4} \cdot \delta \chi.}
\end{equation}

\subsubsection{Másodrendű feszültségtenzor eliminálása, megoldás mátrixos alakban}
\Cref{eq:lin_ed_kappa,eq:lin_ed_c} tartalmaznak $\delta\chi$ függést, amit \cref{eq:lin_ed_chi} segítségével eliminálhatunk. A kapott egyenletek a ${\tau _f} = \alpha \mu b\sqrt {{\rho _0}} $ és ${\tau _r} = {\tau _0} - {\tau _f}$ jelöléssel, \cref{eq:lin_ed_rho}-vel együtt:
\begin{gather}\begin{aligned}
  0 &  = \left( {\lambda  + Ak_x^2} \right) \cdot \delta \rho  + i\frac{{{\tau _0}}}{{bG\sqrt {{\rho _0}} }}{k_x} \cdot \delta \kappa  \\ 
  0 &  = i\left( {{\tau _r} - \frac{{{\tau _f}}}{2}} \right)\frac{1}{{bG\sqrt {{\rho _0}} }}{k_x} \cdot \delta \rho  + \left( {\lambda  + Dk_x^2 + 4\pi \frac{{k_x^2 k_y^2}}{{{k^4}}}} \right)\delta \kappa  - i4\pi \beta \frac{{\sqrt {{\rho _0}} }}{b}\frac{{k_x^2{k_y}}}{{{k^2}}} \cdot \delta c \\ 
  0 &  =  i4\pi \frac{{\beta {M_c}}}{{M_0 b\sqrt {{\rho _0}} }}{k_y} \cdot \delta \kappa  + \left[ {\lambda  + \left( {\frac{\alpha _c}{G} + 4\pi {\beta ^2}} \right)\frac{{{M_c}{k^2}}}{{{M_0 b^2}}}} \right] \cdot \delta c, \\ 
\end{aligned}
\end{gather}
illetve mátrixos írásmódban kapjuk, hogy 
\begin{equation*}
\left({\begin{array}{*{20}{c}}
  {\lambda  + Ak_x^2}&{i\frac{{{\tau _0}}}{{bG\sqrt {{\rho _0}} }}{k_x}}&0 \\ 
  {i\left( {{\tau _r} - \frac{{{\tau _f}}}{2}} \right)\frac{1}{{bG\sqrt {{\rho _0}} }}{k_x}}&{\lambda  + Dk_x^2 + 4\pi \frac{{k_x^2 k_y^2}}{{{k^4}}}}&{ - i4\pi \beta \frac{{\sqrt {{\rho _0}} }}{b}\frac{{k_x^2{k_y}}}{{{k^2}}}} \\ 
  0&{ i4\pi \frac{{\beta {M_c}}}{{M_0 b\sqrt {{\rho _0}} }}{k_y} }&{\lambda  + \left( {\frac{\alpha _c}{G} + 4\pi {\beta ^2}} \right)\frac{{M_c}{k^2}}{M_0 b^2}} 
\end{array}} \right)\left( {\begin{array}{*{20}{c}}
  {\delta \rho } \\ 
  {\delta \kappa } \\ 
  {{\delta _c}} 
\end{array}} \right) = 0.
\end{equation*}
Vezessük be az alábbi új jelöléseket a kevesebb paraméterhez!
\[\begin{gathered}
  \begin{array}{*{20}{c}}
  {\frac{{{\tau _f}}}{{bG\sqrt {{\rho _0}} }} = {{\tilde \tau }_f}}&{\frac{{{\tau _r}}}{{bG\sqrt {{\rho _0}} }} = {{\tilde \tau }_r}} 
\end{array} \hfill \\
  \begin{array}{*{20}{c}}
  {\beta \frac{{\sqrt {{\rho _0}} }}{b} = {\beta ^*}}&{\frac{{{M_c}}}{{{\rho _0}{M_0}}} = M_r^*}&{\frac{{{\alpha _c}}}{G}\frac{{{\rho _0}}}{{{b^2}}} = \alpha _r^*}
\end{array} \hfill \\
  T\left( {\mathbf{k}} \right) = 4\pi k_x^2k_y^2/{k^4} \hfill \\ 
\end{gathered} \]
Ekkor:
\begin{equation} \label{eq:matrix_det_to0}
\tcbhighmath[borderline={1pt}{0pt}{black,solid}]{
\left( {\begin{array}{*{20}{c}}
  {\lambda  + Ak_x^2}&{i\left( {{{\tilde \tau }_r} + {{\tilde \tau }_f}} \right){k_x}}&0 \\ 
  {i\left( {{{\tilde \tau }_r} - \frac{{{{\tilde \tau }_f}}}{2}} \right){k_x}}&{\lambda  + Dk_x^2 + T}&{ - i{\beta ^*}T\frac{{{k^2}}}{{{k_y}}}} \\ 
  0&{i4\pi M_r^*{\beta ^*}{k_y}}&{\lambda + \left( {\alpha _r^* + 4\pi {\beta ^{ * 2}}} \right)M_r^*{k^2}} 
\end{array}} \right)\left( {\begin{array}{*{20}{c}}
  {\delta \rho } \\ 
  {\delta \kappa } \\ 
  {{\delta _c}} 
\end{array}} \right) = 0.}
\end{equation}
Ez akkor lehetséges, ha a mátrix determinánsa $0$. Adott paraméterek mellett ez feltételt szab $lambda$ és $\mathbf{k}$ értékeire. A kérdés az, hogy milyen paraméterértékek mellett lesz pozitív $\lambda$  értékű megoldás, és ahol $\lambda$ maximális, ott mennyi $\mathbf{k}$ értéke.

\subsection{A determináns vizsgálata}
\subsubsection{Visszatekintés az oldott atom nélküli esetre}
A CISM-es könyv 131.\ oldalán találjuk az oldott atomok nélküli esetet, ahol csak $\kappa$ és $\rho$ mennyiségek időfejlődését vizsgálják (a $\chi$ fejlődése adott, itt \aref{eq:lin_ed_chi}.\ egyenlet miatt, oldott atom nélkül pedig a $\delta c$ tag nélkül). A megoldandó feladat ott az, hogy \aref{eq:matrix_det_to0}.\ mátrix jobb felső $2 \times 2$-es részének legyen 0 a determinánsa, azaz
\[\left( {\lambda  + Ak_x^2} \right)\left( {\lambda  + Dk_x^2 + T} \right) + k_x^2B = 0,\]
ahol $B$ a deformáció sebességének egy függvénye,
\begin{equation} \label{eq:B_def}
B = \left( {{{\tilde \tau }_r} + {{\tilde \tau }_f}} \right)\left( {{{\tilde \tau }_r} - {{\tilde \tau }_f}/2} \right).
\end{equation}
Ebből kifejezve a lehetséges két gyököt:

\begin{equation} \label{eq:lambda_k}
{\lambda _ \pm }\left( {\mathbf{k}} \right) =  - \underbrace {\frac{{\left( {A + D} \right)k_x^2 + T\left( {\mathbf{k}} \right)}}{2}}_E \pm \frac{{\sqrt {{{\left[ {\left( {A + D} \right)k_x^2 + T\left( {\mathbf{k}} \right)} \right]}^2} - 4k_x^2\left[ {B  + A\left( {Dk_x^2 + T\left( {\mathbf{k}} \right)} \right)} \right]} }}{2}.
\end{equation}
A nagyobbik gyököt akkor kapjuk, ha a $+$ előjeles megoldást választjuk. A gyök előtti tagot $E$-nek elnevezve kapjuk, hogy
\[{\lambda _ + } =  - E \pm \frac{{\sqrt {{E^2} - 4k_x^2\left[ {B  + A\left( {Dk_x^2 + T} \right)} \right]} }}{2}.\]
Ebből látható, hogy
\[{\lambda _ + } > 0 \Leftrightarrow B  + A\left( {Dk_x^2 + T} \right) < 0.\]
$T$ pozitív, és ha $A$ és $D$ is, akkor $B$ kell, hogy negatív legyen. $B=-1$, $A=D=1$ esetén ábrázolva \aref{eq:lambda_k}.\ egyenlet ${\lambda _ + }\left( {\mathbf{k}} \right)$ függvényét, láthatjuk, hogy mely $\mathbf{k}$ értékek mellett lesz pozitív $\lambda$ megoldás.


Az alábbi ábrán és videón látható az átmenet a lassú (de nem 0) deformációs sebességtől a gyorsig, $B$ függvényében (lásd.\ \ref{eq:B_def}).
\begin{figure}[htb]
\centering
\includemedia[label=def_speed,
addresource=video/def_speed.flv,
activate=pageopen,
passcontext,
flashvars={source=video/def_speed.flv&loop=true}]{\includegraphics[width=0.5\textwidth]{video/def_speed.png}}{VPlayer.swf}
\caption{Videó különböző deformációs sebességekről}
\end{figure}

\begin{figure}[htb]
\centering
\begin{subfigure}[t]{0.495\linewidth}
\centering\includegraphics[scale=0.95]{"figs/lambda_k_surface_B=0"}
\caption{Kritikusan lassú deformációs sebesség}
\end{subfigure}
\begin{subfigure}[t]{0.495\linewidth}
\centering\includegraphics[scale=0.95]{"figs/lambda_k_surface_B=-1"}
\caption{Standard deformációs sebesség}
\end{subfigure}
\begin{subfigure}[t]{0.495\linewidth}
\centering\includegraphics[scale=0.95]{"figs/lambda_k_surface_B=-3_16"}
\caption{Kritikusan gyors deformációs sebesség}
\end{subfigure}
\begin{subfigure}[t]{0.495\linewidth}
\centering\includegraphics[scale=0.95]{"figs/lambda_k_surface_B=-4"}
\caption{Még gyorsabb deformációs sebesség}
\end{subfigure}
\caption{Különböző deformációs sebeség mellett a $\lambda_+$ hullámszám függése}
\end{figure}
\FloatBarrier

\subsection{Oldott atomos általános esete}
\Aref{eq:matrix_det_to0} egyenlet mátrixa $\lambda$ nélkül legyen $M$. Ekkor a keresett determináns:
\[\det \left( {M + \lambda I} \right) = \underbrace {\left( {{M_{11}} + \lambda } \right) \cdot \left( {{M_{22}} + \lambda } \right) \cdot \left( {{M_{33}} + \lambda } \right)}_{P\left( {{\lambda ^3}} \right)} - \underbrace {\left( {{M_{11}} + \lambda } \right){M_{23}}{M_{32}}}_{P\left( \lambda  \right)} - \underbrace {{M_{12}}{M_{21}}\left( {{M_{33}} + \lambda } \right)}_{P\left( \lambda  \right)}.\]
A determináns néhány tulajdonsága.
\begin{enumerate}
\item A komplex mátrixelemek ellenére csak valós együtthatókat tartalmaznak, mert a komplex $M_{12}$ és $M_{21}$-nek csak a szorzata jelenik meg, akárcsak $M_{23}$ és $M_{32}$ esetében.
\item A $M_{23}$ és $M_{32}$ elemeinek a szorzatából kiesik $k_y$, tehát $k_y$ magában nem jelenik meg.
\item Egy valós harmadfokú polinomnak mindig van legalább 1 valós gyöke.
\end{enumerate}
Az általános eset haszontalanul hosszú.

\subsection{Oldott atomos speciális esetei}
Valamilyen speciális esetben még van esély a megoldásra. Ilyen speciális eset lehet:

\begin{enumerate}
\item Ha az oldott atomok sokkal mobilisabbak, mint a diszlokációk. Ekkor igazából \aref{eq:chi}.\ egyenletének $\frac{{\delta {P_s}}}{{\delta \chi }} = 0$ feltételéhez hasonlóan \aref{eq:c_pot}. egyenlet képletét is módosítani kell. Ha jól érzem (de Groma szerint nem), ekkor $M_{33}$ értéke nagy lesz, így $M_{33} > \lambda$ miatt a harmadrendű tagból másodrendű tag lesz, mert a mátrix $3,3$ elemében eltűnik a $\lambda$. Ez a közelítés a valóságban jó eséllyel működik ott, ahol a diszlokációk nem szakadnak ki az oldott atomok felhőjéből.

\item Előírjuk, hogy az oldott atom koncentrációja az előjeles diszlokációsűrűség koncentrációjával kell arányos legyen. Ez azt jelenti, hogy az oldott atom azonnal leköveti a diszlokáció mozgását.

\item Kis koncentrációjú határeset: ekkor lehetne vizsgálni perturbatívan, hogy az oldott atomok nélkül kialakuló esethez képest hogyan változik a hullámhossz. Ez akkor is alkalmazható, ha az oldott atomok mobilitása nem nagyobb sokkal a diszlokációkénál.
\end{enumerate}

\subsection{Megoldás gyors diffúzió esetén}
Tekintve, hogy \aref{eq:matrix_det_to0}.\ egyenlet általános megoldása bonyolult, azt az esetet vizsgáljuk, amikor az oldott atomok diffúziója elég gyors. Ekkor \aref{eq:c_pot}.\ és \aref{eq:c}.\ egyenletek helyett a $\frac{{\delta {P_s}[c]}}{{\delta c}} = \mu_c$ feltételből azt kapjuk, hogy
\begin{equation} \label{eq:fd_c}
\tcbhighmath[borderline={1pt}{0pt}{blue,solid}]{
\mu_c = {\alpha _c} \cdot \left( {c - {c_\infty }} \right) - \beta  \cdot \Delta \chi.}
\end{equation}
A $\rho$-ra és $\kappa$-ra felírt mozgásegyeneltek nem változnak.
\subsubsection{Mozgásegyenletek megoldása lineáris stabilitáselemzéssel}
\Aref{eq:eom}.\ egyenlethez képest a módosítás az, hogy \aref{eq:c}.\ egyenlet helyett \aref{eq:fd_c}.\ egyenletet kell használni, így a megoldandó mozgásegyenletek:
\begin{gather}
\tcbhighmath[borderline={1pt}{0pt}{black,solid}]{
\begin{aligned}
0 & = - \frac{{1 - \nu }}{{2\mu }}{\Delta ^2}\chi  + b\cdot\partial_y\kappa - \beta  \cdot \Delta c \\
  \mu_c & = {\alpha _c} \cdot \left( {c - {c_\infty }} \right) - \beta  \cdot \Delta \chi \\ 
    {\partial _t}\rho  &  =  - {M_0}b{\partial _x}\left\{ {\kappa {\tau _{{\text{mf}}}} - \frac{\kappa }{\rho }GbD{\partial _x}\kappa  - GbA{\partial _x}\rho } \right\} \\ 
  {\partial _t}\kappa  &  =  - {M_0}b{\partial _x}\left\{ {\rho {\tau _{{\text{mf}}}} - \rho \alpha \mu b\sqrt \rho - GbD{\partial _x}\kappa  - \frac{\kappa }{\rho } GbA{\partial _x}\rho } \right\}
\end{aligned}  }
\end{gather}
Mivel a potenciált \aref{eq:plastic_pot_wo_core_reg}.\ egyenletben már úgy definiáltuk, hogy a koncentráció csak a $\chi$-hez csatolódik, nem meglepő, hogy $\rho$ és $\kappa$ mozgásegyenleteiben csak $\chi$-n keresztül jelenik meg. Már itt elvégezhetnénk az első két egyenlet összevonását, de ez már le van írva a könyvben, így inkább előbb elvégzem a lineáris stabilitáselemzést.

Ugyanúgy hullámmegoldásokat keresve egy homogén megoldás körül ugyanazokat az egyenleteket kapjuk $\chi$ alakjára, valamint $\rho$ és $\kappa$ időfejlődésére, mint amit \aref{eq:lin_ed_chi}, \aref{eq:lin_ed_rho}.\ és \aref{eq:lin_ed_kappa}.\ egyenletekben láthattunk. Különbség csak a koncentrációban lesz, \aref{eq:lin_ed_c}.\ egyenleteben, amelyet most a
\[\begin{aligned}
0 & = {\alpha _c}\left( {{c_{\infty} } + \delta {c_0} \cdot {e^{\ldots} } - {c_{\infty}}} \right) - \beta \Delta \left( {xy \cdot {\tau _0} + \delta {\chi _0} \cdot {e^{\ldots} }} \right)\\
   &  = {\alpha _c}\delta {c_0} - \beta \delta {\chi _0} \cdot \left( { - {\rho _0}{k^2}} \right) \\ 
\end{aligned}\]
egyenletből számolva kapjuk, hogy
\begin{equation} \label{eq:fd_c_elim}
\tcbhighmath[borderline={1pt}{0pt}{blue,solid}]{\delta {c} =  - \frac{{\beta {\rho _0}}}{{{\alpha _c}}}{k^2} \cdot \delta {\chi}.}
\end{equation}
Ebben felhasználtuk, hogy a koncentrációban homogén megoldásból következik, hogy a kémiai potenciál 0, mert
\[{\mu _c} = {\alpha _c}\underbrace {\left( {c - {c_\infty }} \right)}_0 - \beta \underbrace {\Delta \chi }_0 = 0.\]
Idemásolva a $\chi$-re vonatkozó \told\ref{eq:lin_ed_chi}+as{}, $\rho$-ra vonatkozó \told\ref{eq:lin_ed_rho}+as{} és $\kappa$-ra vonatkozó \told\ref{eq:lin_ed_kappa}+as{} egyenleteit:
\begin{align*}
\delta \chi & = i\frac{{4\pi Gb}}{{\rho _0^{3/2}}}\frac{{{k_y}}}{{{k^4}}} \cdot \delta \kappa  + \frac{{4\pi G}}{{{\rho _0}}}\beta \frac{1}{{{k^2}}} \cdot \delta c,\\
0 & = \left( {\lambda  + Ak_x^2} \right) \cdot \delta \rho  + i\frac{{{\tau _0}}}{{bG\sqrt {{\rho _0}} }}{k_x} \cdot \delta \kappa,\\
0 & = \lambda \delta\kappa  + i\frac{{{\tau _0}}}{{bG\sqrt {{\rho _0}} }}{k_x} \cdot \delta \rho  - i\frac{{\rho _0^{3/2}}}{{bG}}k_x^2{k_y} \cdot \delta \chi  - i\frac{3}{2}\frac{{\alpha \mu b\sqrt {{\rho _0}} }}{{Gb\sqrt {{\rho _0}} }}{k_x} \cdot \delta \rho  + Dk_x^2 \cdot \delta \kappa.
\end{align*}
Behelyettesítve \aref{eq:fd_c_elim}.\ egyenletből $\delta\chi$ egyenletébe, eliminálhatjuk a maradék kettőből $\delta\chi$-t.
\[\begin{aligned}
  \delta \chi  &  = i\frac{{4\pi Gb}}{{\rho _0^{3/2}}}\frac{{{k_y}}}{{{k^4}}} \cdot \delta \kappa  - \frac{{4\pi G}}{{{\rho _0}}}\beta \frac{1}{{{k^2}}} \cdot \frac{{\beta {\rho _0}}}{{{\alpha _c}}}{k^2}\delta \chi  \\ 
   &  = i\frac{{4\pi Gb}}{{\rho _0^{3/2}}}\frac{1}{{1 + 4\pi G\frac{{{\beta ^2}}}{{{\alpha _c}}}}}\frac{{{k_y}}}{{{k^4}}} \cdot \delta \kappa  \\ 
\end{aligned} \]
Láthatjuk, hogy ugyanaz a tag jön be, mint ami a CISM könyv 74-es egyenletében. \Aref{eq:lin_ed_kappa}.\ egyenletben, vagyis a $\kappa$-ra vonatkozó egyenletben van csak $\chi$ függés, így csak azt kell módosítani, a $\rho$-ra vonatkozó egyenletet pedig nem módosítani. Mivel a koncentrációt kifejezhetjük $\chi$-vel, így a $\kappa$ teljes, új módosított mozgásegyenletében már nem lesz koncentráció. A módosított tag:
\[\begin{aligned}
   - \frac{{i\rho _0^{3/2}k_x^2{k_y}}}{{bG}} \cdot \delta \chi  &  =  - \frac{{i\rho _0^{3/2}k_x^2{k_y}}}{{bG}} \cdot i\frac{{4\pi Gb}}{{\rho _0^{3/2}}}\frac{1}{{1 + 4\pi G\frac{{{\beta ^2}}}{{{\alpha _c}}}}}\frac{{{k_y}}}{{{k^4}}} \cdot \delta \kappa  \\ 
   &  = \frac{1}{{1 + 4\pi G\frac{{{\beta ^2}}}{{{\alpha _c}}}}}4\pi \frac{{k_x^2k_y^2}}{{{k^4}}}\delta \kappa  \\ 
   &  = \frac{1}{{1 + 4\pi G\frac{{{\beta ^2}}}{{{\alpha _c}}}}}T\left( {\mathbf{k}} \right)\delta \kappa.  \\ 
\end{aligned} \]
Így bevezetve a
\[{T_c}\left( {\mathbf{k}} \right) = \frac{1}{{1 + \tilde \beta^2}}T\left( {\mathbf{k}} \right)\quad \tilde \beta^2 = 4\pi G\frac{{{\beta ^2}}}{{{\alpha _c}}}\]
jelölést, ugyanazt kapjuk, mint az oldott atom nélküli esetben, csak $T$ helyett $T_c$-vel.
\begin{equation*}
\left({\begin{array}{*{20}{c}}
  {\lambda_c  + Ak_x^2}&{i\frac{{{\tau _0}}}{{bG\sqrt {{\rho _0}} }}{k_x}} \\ 
  {i\left( {{\tau _r} - \frac{{{\tau _f}}}{2}} \right)\frac{1}{{bG\sqrt {{\rho _0}} }}{k_x}}&{\lambda_c  + Dk_x^2 + {T_c}\left( {\mathbf{k}} \right)}
\end{array}} \right)\left( {\begin{array}{*{20}{c}}
  {\delta \rho } \\ 
  {\delta \kappa }
\end{array}} \right) = 0.
\end{equation*}
Ez nem változtatja meg lényegesen \aref{eq:lambda_k}.\ egyenletet, amelynek most az új alakja

\begin{equation} \label{eq:lambda_c_k}
{\lambda _{c, + }}\left( {\mathbf{k}} \right) =  - \frac{{\left( {A + D} \right)k_x^2 + {T_c}\left( {\mathbf{k}} \right)}}{2} + \frac{{\sqrt {{{\left[ {\left( {A + D} \right)k_x^2 + {T_c}\left( {\mathbf{k}} \right)} \right]}^2} - 4k_x^2\left[ {B  + A\left( {Dk_x^2 + {T_c}\left( {\mathbf{k}} \right)} \right)} \right]} }}{2}.
\end{equation}
Beszorozva $T/T_c$-vel, láthatjuk, hogy $\lambda_{c,+}$, $A$ és $D$ elsőfokon, $B$ másodfokon átskálázódik. Tehát az új $A$, $D$ és $B$ paraméterek függvényében minden ugyanolyan alakú. Alább látható egy videó és 4 jellemző értéke mellett az átmenet a ${\lambda _{c, + }}\left( {\mathbf{k}} \right)$ térben a 0 erősségű csatolástól az erős csatolásig.

\begin{figure}[htb]
\centering
\includemedia[label=gyors_diff_reg,
addresource=video/fast_diff.flv,
activate=pageopen,
passcontext,
flashvars={source=video/fast_diff.flv&loop=true}]{\includegraphics[width=0.5\textwidth]{video/fast_diff.png}}{VPlayer.swf}
\caption{Videó, erősödő oldott atom kh.\ csatolással}
\end{figure}

\begin{figure}[htb]
\centering
\begin{subfigure}[t]{0.495\linewidth}
\centering\includegraphics[scale=0.95]{"figs/lambda_c_k_surface_btsq=0"}
\caption{$\tilde \beta^2=0$, nincs csatolás}
\end{subfigure}
\begin{subfigure}[t]{0.495\linewidth}
\centering\includegraphics[scale=0.95]{"figs/lambda_c_k_surface_btsq=1"}
\caption{$\tilde \beta^2=1$, gyenge csatolás}
\end{subfigure}
\begin{subfigure}[t]{0.495\linewidth}
\centering\includegraphics[scale=0.95]{"figs/lambda_c_k_surface_btsq=10"}
\caption{$\tilde \beta^2=10$, közepes csatolás}
\end{subfigure}
\begin{subfigure}[t]{0.495\linewidth}
\centering\includegraphics[scale=0.95]{"figs/lambda_c_k_surface_btsq=100"}
\caption{$\tilde \beta^2=100$, erős csatolás}
\end{subfigure}
\caption{Különböző erősséggel csatolva az oldott atomokat a rendszerhez, változik a ${\lambda _{c, + }}\left( {\mathbf{k}} \right)$ tér.}
\end{figure}
\FloatBarrier



\section{Core regularizáció oldott atom nélkül}
Hogy megértsük az oldott atom esetét core regularizációval, érdemes először megnézni oldott atom nélkül, de core regularizációval a diszlokációmintázat-képződést. Ehhez \aref{eq:plastic_pot}.\ egyenletben definiált plasztikus potenciál oldott atomos tagjait elhanyagolva kapjuk, hogy 
\begin{equation} \label{eq:plastic_pot_dc}
{P_{dc}} = \int { - \frac{{1 - \nu }}{{4\mu }}{{\left( {\Delta \chi } \right)}^2} + b\chi \frac{{\partial \kappa }}{{\partial y}} + {a^2}{{\left| {\nabla \left( {{\text{ }}\Delta \chi } \right)} \right|}^2} + G{b^2}A\rho \ln \left( {\frac{\rho }{{{\rho _0}}}} \right) + \frac{{G{b^2}D}}{2}\frac{{{\kappa ^2}}}{\rho }{d^2}r}.
\end{equation}
Itt ${P_{dc}}\left[ {\chi ,\rho ,\kappa } \right]$. Ennek a funkcionális deriváltjaiból ugyanazokat az egyenleteket kapjuk $\kappa$ és $\rho$ időfejlődésére, mint \aref{eq:rho_kappa}.\ egyenletben. Az eltérés $\chi$-ben van, ugyanis itt \aref{eq:chi}.\ egyenlettel szemben, a $\frac{{\delta {P_{dc}}}}{{\delta \chi }} = 0$ feltételből azt kapjuk, hogy 
\[ - \frac{{1 - \nu }}{{2\mu }}{\Delta ^2}\chi  + b\frac{{\partial \kappa }}{{\partial y}} + 2{a^2}{\Delta ^3}\chi  = 0.\]
Így tehát most az alábbi egyenleteket kell megoldanunk:

\begin{gather} \label{eq:eom_dc}
\tcbhighmath[borderline={1pt}{0pt}{black,solid}]{
\begin{aligned}
0 & = - \frac{{1 - \nu }}{{2\mu }}{\Delta ^2}\chi  + b\frac{{\partial \kappa }}{{\partial y}} + 2{a^2}{\Delta ^3}\chi  \\ 
    {\partial _t}\rho  &  =  - {M_0}b{\partial _x}\left\{ {\kappa {\tau _{{\text{mf}}}} - \frac{\kappa }{\rho }GbD{\partial _x}\kappa  - GbA{\partial _x}\rho } \right\} \\ 
  {\partial _t}\kappa  &  =  - {M_0}b{\partial _x}\left\{ {\rho {\tau _{{\text{mf}}}} - \rho \alpha \mu b\sqrt \rho - GbD{\partial _x}\kappa  - \frac{\kappa }{\rho } GbA{\partial _x}\rho } \right\}.\\
\end{aligned}  }
\end{gather}
\subsection{Lineáris stabilitáselemzés}
A korábbi receptet követve keressük az egyes mennyiségeket
\[f\left( {{\mathbf{r}},t} \right) = {f_0} + \delta f\left( {{\mathbf{r}},t} \right) = {f_0} + \delta {f_0} \cdot \exp \left( {\frac{\lambda }{{{t_0}}}t + i\sqrt {{\rho _0}}  \cdot {\mathbf{kr}}} \right)\]
alakban. Az operátorok felveszik \aref{eq:sajat_ertek_op}.\ egyenletben leírt sajátértékeiket, valamint van egy új operátorunk, amelyre
\[{\Delta ^3} \to {\left( {i\sqrt {{\rho _0}} } \right)^6} \cdot {\left( {k_x^2 + k_y^2} \right)^3} =  - \rho _0^3 \cdot {k^6}.\]
Ekkor \aref{eq:eom_dc}.\ egyenlet $\chi$-re vonatkozó egyenletéből kifejezhetjük $\delta \chi$-t:
\[
  0  =  - \frac{{1 - \nu }}{{2\mu }}\rho _0^2 \cdot {k^4} \cdot \delta \chi  + b \cdot i\sqrt {{\rho _0}}  \cdot {k_y} \cdot \delta \kappa  - 2{a^2}\rho _0^3 \cdot {k^6} \cdot \delta \chi,\]
amelyből 
\begin{equation} \label{eq:lin_ed_chi_dc}
\tcbhighmath[borderline={1pt}{0pt}{blue,solid}]{
\delta \chi  = i\frac{{b\sqrt {{\rho _0}} }}{{\frac{{1 - \nu }}{{2\mu }}\rho _0^2 \cdot {k^4} + 2{a^2}\rho _0^3 \cdot {k^6}}}{k_y} \cdot \delta \kappa.}
\end{equation}
A teljes diszlokációsűrűség időfejlődését  \aref{eq:lin_ed_rho}.\ egyenlet szerint egyáltalán nem érinti $\chi$,
\begin{equation*}
\tcbhighmath[borderline={1pt}{0pt}{blue,solid}]{
0 = \left( {\lambda  + Ak_x^2} \right) \cdot \delta \rho  + i\frac{{{\tau _0}}}{{bG\sqrt {{\rho _0}} }}{k_x} \cdot \delta \kappa, }
\end{equation*}
és az előjeleset is csak $\delta \chi$-n keresztül \aref{eq:lin_ed_kappa}.\ egyenlet szerint:
\[0 = \lambda \delta \kappa  + i\left( {\frac{{{\tau _0}}}{{bG\sqrt {{\rho _0}} }} - \frac{3}{2}\frac{{\alpha \mu }}{G}} \right){k_x} \cdot \delta \rho  - i\frac{{\rho _0^{3/2}}}{{bG}}k_x^2{k_y} \cdot \delta \chi  + Dk_x^2 \cdot \delta \kappa .\]
Behelyettesítve ebbe \aref{eq:lin_ed_chi_dc}.\ egyenlettel, kapjuk, hogy 
\begin{equation} \label{eq:lin_ed_kappa_dc}
\tcbhighmath[borderline={1pt}{0pt}{blue,solid}]{
0 = \lambda \delta \kappa  + i\left( {\frac{{{\tau _0}}}{{bG\sqrt {{\rho _0}} }} - \frac{3}{2}\frac{{\alpha \mu }}{G}} \right){k_x}\delta \rho  + {T_{dc}} \cdot \delta \kappa  + Dk_x^2 \cdot \delta \kappa,}
\end{equation}
amelyben bevezettem a 
\[{T_{dc}}\left( {\mathbf{k}},{\tilde a}^2 \right) = \frac{{4\pi  \cdot k_x^2k_y^2}}{{{k^4} + {{\tilde a}^2} \cdot {k^6}}}\quad {{\tilde a}^2} = 8\pi {a^2}G{\rho _0}\]
jelöléseket, és így ugyanolyan mátrixegyenletet kell megoldani, mint a CISM könyvben, egészen pontosan $\lambda$-ra azt kapjuk, hogy 
\[{\lambda _{dc, + }} =  - \frac{{\left( {A + D} \right)k_x^2 + {T_{dc}}}}{2} + \frac{{\sqrt {{{\left[ {\left( {A + D} \right)k_x^2 + {T_{dc}}} \right]}^2} - 4k_x^2\left[ {B + A\left( {Dk_x^2 + {T_{dc}}} \right)} \right]} }}{2}.\]

\subsection{Regularizáció erősségének hatása a kiválasztott hullámhosszon}
\paragraph{Ha $\tilde a$ értéke közepes}, akkor érdekes a jelenség igazán. A $k \in \left[ {0,1} \right]$ tartományon ${k^4} \approx {\tilde a^2}{k^6} \Leftrightarrow {\tilde a^2} \in \left[ {1,100} \right]$.

\paragraph{Ha $\tilde a$ kicsi}, akkor ${T_{dc}} = T$, vagyis visszakapjuk a core regularizáció nélküli esetét. Ez logikus is, hiszen a core hatását elhanyagoltuk ekkor. 

\paragraph{Ha $\tilde a$ nagy}, akkor a ${T_{dc}}$ értéke elhanyagolható lesz, ekkor pedig a
\[\mathop {\lim }\limits_{\tilde a \to \infty } {\lambda _{dc, + }} =  - \frac{{\left( {A + D} \right)k_x^2}}{2} + \frac{{\sqrt {{{\left[ {\left( {A + D} \right)k_x^2} \right]}^2} - 4k_x^2\left[ {B + A\left( {Dk_x^2} \right)} \right]} }}{2}\]
egyenletet kell megoldani.

\end{document}


