 \documentclass[10pt,a4paper]{scrartcl}

\usepackage{graphicx}
\usepackage{amsmath}
\usepackage{amsfonts}
\usepackage{amssymb}
\usepackage{bm}
\let\mathbf\bm
\usepackage[left=2.5cm, right=3cm, top=2cm]{geometry}

\usepackage[T1]{fontenc}
\usepackage[utf8]{inputenc}
\usepackage[magyar]{babel}
\usepackage{lmodern}

\usepackage{placeins}
\usepackage{subcaption}
\usepackage{epstopdf}
\usepackage{xcolor}
\usepackage[hidelinks,unicode]{hyperref}
\hypersetup{
    colorlinks,
    linkcolor={red!50!black},
    citecolor={blue!50!black},
    urlcolor={blue!80!black}
}
\usepackage[most]{tcolorbox}
\tcbset{highlight math style={
	enhanced,
	boxrule=0pt,
	colframe=blue,
	colback=white,
	arc=0pt,
	boxrule=0pt,
	left={3pt},
	right={3pt},
	top={3pt},
	bottom={3pt}}}

\usepackage{cleveref}
\newcommand*\Laplace{\mathop{}\!\mathbin\bigtriangleup}

\begin{document}
\tableofcontents

\section{Ismertető}
Szeretnénk egy olyan funkcionált felírni, aminek a megfelelő funkcionálderiváltjaiból kijönnek a megfelelő mozgásegyenletek. A funkcionál függvénye az alábbiaknak:
\[P=P\left[ {\chi ,{\rho _ + },{\rho _ - },c} \right].\]
Úgy képezzük az eredő potenciált, hogy összeadjuk a rugalmas közeg - diszlokációk önkonzisztens potenciálját, a diszlokációk mintázatképződésért felelős potenciált, illetve az oldott atomok önkölcsönhatásból és a nyomással való kölcsönhatásból származtatott potenciálokat.
\[P = {P_{sc}} + {P_{corr}} + {P_{c - cp}}\]
A $P_{c - cp}$ tag lehet a $c\ln \left( c \right)$ alakú $M_c \sim c$ mobilitással, vagy ${\left( {c - {c_\infty }} \right)^2}$ alakú konstans $M_c$ mobilitással. Core regularizációs tagokat megpróbálhatjuk 0-nak venni, $a' = a = 0$.
\begin{multline} \label{eq:plastic_pot}
P = \int \left[ \underbrace { - \frac{{1 - \nu }}{{4\mu }}{{\left( {\Laplace \chi } \right)}^2} + b\chi \frac{\partial \kappa}{\partial y} + \overbrace{a^2\left| {\nabla \left( \Laplace \chi  \right)} \right|^2}^{\text{core reg.}}}_{{\text{önkonzisztens tér}}} + \underbrace {G{b^2}A\rho \ln \left( {\frac{\rho }{{{\rho _0}}}} \right) + \frac{{G{b^2}D}}{2}\frac{{{\kappa ^2}}}{\rho }}_{{\text{diszlokációk korrelációjához}}} + \right. \\ 
\left. + \underbrace {\frac{\alpha_c}{2} {{\left( {c - {c_\infty }} \right)}^2}}_{\substack{\text{oldott atom}\\\text{önkh.}}} - \underbrace {\beta c\left( {\Laplace \chi } \right)}_{\substack{\text{oldott atom}\\\text{nyomás kh-a}}} + \overbrace {a{'^2}\nabla \left( \Laplace \chi  \right) \cdot \nabla c}^{{\text{core reg.}}}\right] {d^2}r.
\end{multline}
Az egyenlet $\left[ {\chi ,{\rho _ + },{\rho _ - },c} \right]$-nak a függvénye, és az egyes argumentumok szerinti parciális derviáltakból 1-1 egyenletet kapunk. Sorrendben ezek az alábbiak.
\subsection{Másodrendű feszültségtenzor}
A $\chi$ , $\kappa$ és immár $c$ közötti kapcsolatot kifejező egyenlet: 
\begin{equation}
\frac{{\delta P}}{{\delta \chi }} = 0.
\end{equation}
Vagyis feltesszük, hogy a rendszer azonnal beáll abba az állapotba, ahol az elasztikus energiája minimális. Ez jó közelítés, amíg a karakterisztikus sebességek kisebbek a hangsebességnél. Ennek segítségével a többi egyenletből (\cref{eq:rho_eom,eq:kappa_eom,eq:c_conti}) eliminálható $\chi$.
\subsection{Diszlokációsűrűségek} A $\kappa$ és $\rho$ időfejlődésére vonatkozó egyenletek:
\begin{align} \label{eq:rho_eom}
  {\partial _t}\rho  & =  - {\partial _x}\left\{ {\kappa {M_0} \cdot \left( { - {\partial _x}\frac{{\delta P}}{{\delta \kappa }}} \right) + \rho {M_0} \cdot \left( { - {\partial _x}\frac{{\delta P}}{{\delta \rho }}} \right)} \right\} \\
  {\partial _t}\kappa  & =  - {\partial _x}\left\{ {\rho M\left( { - {\partial _x}\frac{{\delta P}}{{\delta \kappa }}} \right)\quad  + \kappa {M_0} \cdot \left( { - {\partial _x}\frac{{\delta P}}{{\delta \rho }}} \right)} \right\}. \label{eq:kappa_eom}
\end{align}
A korábbi cikkekben és a Mesoscale könyveben az előjelekkel egyszerűsítve van, viszont abban az esetben a mobilitás hibásan szerepel. A helyes mobilitást viszont nem lehet felírni intuitív módon pozitív argumentumra. \Cref{eq:rho_eom,eq:kappa_eom}-ben a $\partial_x$ úgy hat a $\kappa /\rho $-ra, hogy csak $\kappa$-ban derivál, mert a magasabbrendű tagokat elhagyjuk. Kicsi $\kappa /\rho $, ${\partial _x}\rho /{\rho ^{3/2}}$ és $\frac{{c - {c_\infty }}}{{{c_\infty }}}$, azaz ezek szorzatait és magasabbrendű hatványait elhagyjuk.

A mobilitás egy nemtriviális függvénye az argumentumának, de mi csak ott vizsgálódunk, ahol 
\begin{equation} \label{eq:nontriv_mobility}
M\left( { - {\partial _x}\frac{{\delta P}}{{\delta \kappa }}} \right) = {M_0}\left[ { - {\partial _x}\frac{{\delta P}}{{\delta \kappa }} - \alpha \mu {b^2}\sqrt \rho} \right].
\end{equation}
Ebben már elhanyagoltunk egy $ \kappa ^2 / \rho ^2 $-es tagot.

\subsection{Oldott atom koncentrációja} Van egy egyenletünk $c$ megmaradására:
\begin{equation} \label{eq:c_conti}
\partial_t c =  \nabla \left( {{M_c} \cdot \nabla \frac{{\delta P}}{{\delta c}}} \right).
\end{equation}

\subsection{Megoldási terv}
Az \cref{eq:rho_eom,eq:kappa_eom,eq:c_conti}-en lineáris stabilitáselemzést lehet végezni, ami egy $3\times3$-as mátrix vizsgálatához vezet: a mátrix determinánsa legyen 0.



\section{Megoldás core regularizáció nélkül}
\subsection{A mozgásegyeneltek felírása}
Nézzük meg, hogy core regularizáció nélkül mire jutunk. A diszlokációk korrelációjából származó tagokat is tartalmaz ez a potenciál, így reméljük, hogy hátha megfelelően elkeni a core szinguláris problémáját. A potenciált tehát $a=0$ és $a'=0$ mellett:
\begin{equation} \label{eq:plastic_pot_wo_core_reg}
P_s = \int { \underbrace{- \frac{{1 - \nu }}{{4\mu }}{{\left( {\Laplace\chi } \right)}^2}}_{P_{s,1}} + \underbrace{b\chi \frac{{\partial \kappa }}{{\partial y}}}_{P_{s,2}} + \underbrace{G{b^2}A\rho \ln \left( {\frac{\rho }{{{\rho _0}}}} \right)}_{P_{s,3}} + \underbrace{\frac{{G{b^2}D}}{2}\frac{{{\kappa ^2}}}{\rho }}_{P_{s,4}} + \underbrace{\frac{\alpha_c}{2} {\left( {c - {c_\infty }} \right)^2}}_{P_{s,5}} + \underbrace{\beta c\left(- {\Laplace\chi } \right)}_{P_{s,6}}}{d^2}r.
\end{equation}
\subsubsection{Másodrendű feszültségtenzor}
A $\frac{{\delta {P_s}}}{{\delta \chi }} = 0$ feltételből:
\begin{equation} \label{eq:chi}
\tcbhighmath[borderline={1pt}{0pt}{blue,solid}]{- \frac{{1 - \nu }}{{2\mu }}{\Delta ^2}\chi  + b\frac{{\partial \kappa }}{{\partial y}} - \beta  \cdot \Delta c = 0.}
\end{equation}
\subsubsection{Diszlokáció sűrűségek} $\kappa$ kontinuitási egyenletéhez \cref{eq:kappa_eom} egyenletbe helyettesítsünk be az $M$ mobilitásával a \cref{eq:nontriv_mobility}-val, $\rho$-é változatlan:
\begin{gather} \label{eq:rho_kappa_conti}
\tcbhighmath[borderline={1pt}{0pt}{blue,dashed}]{
\begin{aligned}{\partial _t}\rho  &  = {M_0}{\partial _x}\left\{ {\kappa \cdot \left( {{\partial _x}\frac{{\delta P}}{{\delta \kappa }}} \right) + \rho \cdot {\partial _x}\frac{{\delta P}}{{\delta \rho }}} \right\}, \\ 
  {\partial _t}\kappa  &  = {M_0}{\partial _x}\left\{ {\rho \left[ {{\partial _x}\frac{{\delta P}}{{\delta \kappa }} + \alpha \mu {b^2}\sqrt \rho } \right] + \kappa  \cdot {\partial _x}\frac{{\delta P}}{{\delta \rho }}} \right\}.
\end{aligned}}
\end{gather}
Ennek megadásához el kell végezni a $\frac{{\delta P}}{{\delta \rho }}$ és $\frac{{\delta P}}{{\delta \kappa }}$ deriválást. Először $\kappa$-ra az egyes tagokon elvégezve:
\begin{align*}
  {P_{s,2}}\left[ {\kappa  + \delta \kappa } \right] &  = \int_{{L^2}} {b\chi \left( {\frac{\partial }{{\partial y}}\left( {\kappa  + \delta \kappa } \right)} \right)dA}  \\ 
   &  = \underbrace {\int_{{L^2}} {b\chi \left( {\frac{\partial }{{\partial y}}\kappa } \right)dA} }_{{P_{s,2}}\left[ \kappa  \right]} + \int_{{L^2}} {b\chi \left( {\frac{\partial }{{\partial y}}\delta \kappa } \right)dA}  \\ 
   &  = {P_{s,2}}\left[ \kappa  \right] + \int_{{L^2}} {\int\limits_{ - \infty }^\infty  {b\chi \left( {\frac{\partial }{{\partial y}}\delta \kappa } \right)} dy} dx \\ 
   &  = {P_{s,2}}\left[ \kappa  \right] + \int_{{L^2}} {\underbrace {\int\limits_{ - \infty }^\infty  {b\left( {\frac{\partial }{{\partial y}}\left( {\chi  \cdot \delta \kappa } \right)} \right)} dy}_{\left. {\chi  \cdot \delta \kappa } \right|_{ - \infty }^\infty  = 0}} dx - \int_{{L^2}} {\int\limits_{ - \infty }^\infty  {b\left( {\frac{\partial }{{\partial y}}\left( \chi  \right) \cdot \delta \kappa } \right)dy} dx}  \\ 
   &  = {P_{s,2}}\left[ \kappa  \right] - \int_{{L^2}} {b\left( {\frac{\partial }{{\partial y}}\left( \chi  \right) \cdot \delta \kappa } \right)dA}, \text{ tehát}
\end{align*}
\begin{equation} \label{eq:ps2_kappa}
{P_{s,2}}\left[ {\kappa  + \delta \kappa } \right] - {P_{s,2}}\left[ \kappa  \right] =  - \int_{{L^2}} {b\left( {\frac{\partial }{{\partial y}}\left( \chi  \right) \cdot \delta \kappa } \right)dA}  \Rightarrow \tcbhighmath[borderline={0.5pt}{0pt}{blue,dotted}]{ \frac{{\delta {P_{s,2}}\left[ \kappa  \right]}}{{\delta \kappa }} =  - b\frac{{\partial \chi }}{{\partial y}}}
\end{equation}
\begin{equation} \label{eq:ps4_kappa}
{P_{s,4}}\left[ \kappa  \right] = \int_{{L^2}} {\frac{{G{b^2}D}}{2}\frac{{{\kappa ^2}}}{\rho }dA}  \Rightarrow \tcbhighmath[borderline={0.5pt}{0pt}{blue,dotted}]{\frac{{\delta {P_{s,4}}\left[ \kappa  \right]}}{{\delta \kappa }} = G{b^2}D\frac{\kappa }{\rho }}
\end{equation}
A \cref{eq:ps2_kappa,eq:ps4_kappa} egyenletekből
\begin{equation} \label{eq:ps_kappa}
\tcbhighmath[borderline={1pt}{0pt}{blue,dashed}]{\frac{{\delta {P_s}\left[ \kappa  \right]}}{{\delta \kappa }} = G{b^2}D\frac{\kappa }{\rho } - b\frac{{\partial \chi }}{{\partial y}}.}
\end{equation}
Elvégezve $\rho$-ra az egyes tagokon:
\[\begin{aligned}
  {P_{s,3}}\left[ {\rho  + \delta \rho } \right] &  = \int_{{L^2}} {G{b^2}A \cdot \left( {\rho  + \delta \rho } \right)\underbrace {\ln \left( {\left( {\rho  + \delta \rho } \right)/{\rho _0}} \right)}_{\ln \left( {\rho /{\rho _0}} \right) + \delta \rho /\rho }} dA \\ 
   &  = {P_{s,3}}\left[ \rho  \right] + \int_{{L^2}} {G{b^2}A \cdot \delta \rho  \cdot \ln \left( {\rho /{\rho _0}} \right)} dA + \int_{{L^2}} {G{b^2}A \cdot \delta \rho } dA, \\ 
\end{aligned} \]
amelyből egyrészt
\begin{equation} \label{eq:ps3_rho}
\tcbhighmath[borderline={0.5pt}{0pt}{blue,dotted}]{\frac{{{P_{s,3}}\left[ \rho  \right]}}{{\delta \rho }} = G{b^2}A \cdot \ln \left( {\rho /{\rho _0}} \right) + G{b^2}A,}
\end{equation}
másrészt
\[\begin{aligned}
  {P_{s,4}}\left[ {\rho  + \delta \rho } \right] &  = \int_{{L^2}} {\frac{{G{b^2}D}}{2} \cdot \underbrace {\frac{{{\kappa ^2}}}{{\rho  + \delta \rho }}}_{\frac{{{\kappa ^2}}}{\rho } - \frac{{{\kappa ^2}}}{{{\rho ^2}}}\delta \rho \approx \frac{\kappa^2}{\rho }}} dA \\ 
   &  = {P_{s,4}}\left[ \rho  \right], \\ 
\end{aligned} \]
amelyből
\begin{equation} \label{eq:ps4_rho}
\tcbhighmath[borderline={0.5pt}{0pt}{blue,dotted}]{\frac{{{P_{s,4}}\left[ \rho  \right]}}{{\delta \rho }} =  0,}
\end{equation}
így \cref{eq:ps3_rho,eq:ps4_rho} egyenleteket felhasználva kapjuk:
\begin{equation} \label{eq:ps_rho}
\tcbhighmath[borderline={1pt}{0pt}{blue,dashed}]{\frac{{{P_s}\left[ \rho  \right]}}{{\delta \rho }} = G{b^2}A \cdot \ln \left( {\rho /{\rho _0}} \right) + G{b^2}A .}
\end{equation}
Be kell helyettesíteni \cref{eq:rho_kappa_conti}-ba \cref{eq:ps_kappa,eq:ps_rho}-vel. Ha ${{\partial _x}\frac{{\delta P}}{{\delta \kappa }}} > 0$, akkor:
\[\begin{aligned}
  {\partial _t}\rho  &  = {M_0}{\partial _x}\left\{ {\kappa  \cdot {\partial _x}\left( {G{b^2}D \cdot \underbrace {\frac{\kappa }{\rho }}_{{\partial _x}\frac{\kappa }{\rho } \approx \frac{1}{\rho }{\partial _x}\kappa } - b\frac{{\partial \chi }}{{\partial y}}} \right) + \rho {\partial _x}\left( {G{b^2}A \cdot \ln \left( {\frac{\rho }{{{\rho _0}}}} \right) + G{b^2}A} \right)} \right\} \\ 
   &  = {M_0}b{\partial _x}\left\{ {\kappa  \cdot \left( {GbD\frac{{{\partial _x}\kappa }}{\rho } - \underbrace {\frac{{{\partial ^2}\chi }}{{\partial x\partial y}}}_{{\tau _{{\text{mf}}}}}} \right) + \rho  \cdot GbA\frac{{{\partial _x}\rho }}{\rho }} \right\}, \\ 
  {\partial _t}\kappa  &  = {M_0}{\partial _x}\left\{ {\rho \left[ {{\partial _x}\left( {G{b^2}D\frac{\kappa }{\rho } - b\frac{{\partial \chi }}{{\partial y}}} \right) + \alpha \mu {b^2}\sqrt \rho } \right] + \kappa {\partial _x}\left( {G{b^2}A\ln \left( {\frac{\rho }{{{\rho _0}}}} \right) + G{b^2}A} \right)} \right\} \\ 
   &  = {M_0}b{\partial _x}\left\{ {\rho \left( {GbD\frac{{{\partial _x}\kappa }}{\rho } - \underbrace {\frac{{{\partial ^2}\chi }}{{\partial x\partial y}}}_{{\tau _{{\text{mf}}}}} + \alpha \mu {b^2}\sqrt \rho } \right) + \kappa  \cdot GbA\frac{{{\partial _x}\rho }}{\rho }} \right\}, \\ 
\end{aligned} \]
vagyis összesen:
\begin{gather} \label{eq:rho_kappa}
\tcbhighmath[borderline={1pt}{0pt}{blue,solid}]{
\begin{aligned}
  {\partial _t}\rho  &  =  - {M_0}b{\partial _x}\left\{ {\kappa {\tau _{{\text{mf}}}} - \frac{\kappa }{\rho }GbD{\partial _x}\kappa  - GbA{\partial _x}\rho } \right\}, \\ 
  {\partial _t}\kappa  &  =  - {M_0}b{\partial _x}\left\{ {\rho {\tau _{{\text{mf}}}} - \rho \alpha \mu b\sqrt \rho - GbD{\partial _x}\kappa  - \frac{\kappa }{\rho } GbA{\partial _x}\rho } \right\}.
  \end{aligned}}
\end{gather}
\subsubsection{Oldott atom koncentráció} A koncentrációra vonatkozó egyenlethez $\frac{{\delta {P_s}\left( c \right)}}{{\delta c}}$-t kell kiszámolni.
\begin{align}
  {P_{s,5}}\left[ {c + \delta c} \right] &  = \int_{{L^2}} {\frac{\alpha_c}{2}  \cdot \underbrace {{{\left( {c + \delta c - {c_\infty }} \right)}^2}}_{{{\left( {c - {c_\infty }} \right)}^2} + 2\left( {c - {c_\infty }} \right) \cdot \delta c + {{\left( {\delta c} \right)}^2}}dA} \nonumber \\ 
   &  = {P_{s,5}}\left[ c \right] + \int_{{L^2}} {\frac{\alpha_c}{2}  \cdot 2\left( {c - {c_\infty }} \right)\delta cdA}  \Rightarrow\tcbhighmath[borderline={0.5pt}{0pt}{blue,dotted}]{ \frac{{\delta {P_{s,5}}\left[ c \right]}}{{\delta c}} = \alpha_c  \cdot \left( {c - {c_\infty }} \right)}
\end{align}
\begin{equation}
{P_{s,6}}\left[ {c + \delta c} \right] = \int_{{L^2}} {\beta \left( {c + \delta c} \right)\left( { - \Delta \chi } \right)} dA \Rightarrow \tcbhighmath[borderline={0.5pt}{0pt}{blue,dotted}]{\frac{{\delta {P_{s,6}}\left[ c \right]}}{{\delta c}} =  - \beta  \cdot \Delta \chi}
\end{equation}
\begin{equation}
\tcbhighmath[borderline={1pt}{0pt}{blue,dashed}]{\frac{{\delta {P_s}\left[ c \right]}}{{\delta c}} = \alpha_c  \cdot \left( {c - {c_\infty }} \right) - \beta  \cdot \Delta \chi }
\end{equation}
A koncentráció kontinuitási egyenletéből megkapjuk az időfejődést:
\[\partial_t c =  - \nabla \left( {{M_c} \cdot \nabla \left( {{\alpha _c} \cdot \left( {c - {c_\infty }} \right) - \beta  \cdot \Delta \chi } \right)} \right),\]
amelyben $M_c$ és $c_{\infty}$ állandó térben, így
\begin{equation} \label{eq:c}
\tcbhighmath[borderline={1pt}{0pt}{blue,solid}]{
\partial_t c =  - {M_c}{\alpha _c}\Delta c + \beta {M_c}\Delta^2 \chi}.
\end{equation}

\subsection{Mozgáegyenletek megoldása lineáris stabilitáselemzéssel}
\Cref{eq:chi,eq:rho_kappa,eq:c} egyenletek együtt, $\partial_x \partial_y \chi = {\tau _{mf}}$ jelöléssel, $\kappa^2/\rho^2$-es tag elhanyagolásával:
\begin{gather}
\tcbhighmath[borderline={1pt}{0pt}{black,solid}]{
\begin{aligned}
0 & = - \frac{{1 - \nu }}{{2\mu }}{\Delta ^2}\chi  + b\cdot\partial_y\kappa - \beta  \cdot \Delta c \\ 
    {\partial _t}\rho  &  =  - {M_0}b{\partial _x}\left\{ {\kappa {\tau _{{\text{mf}}}} - \frac{\kappa }{\rho }GbD{\partial _x}\kappa  - GbA{\partial _x}\rho } \right\} \\ 
  {\partial _t}\kappa  &  =  - {M_0}b{\partial _x}\left\{ {\rho {\tau _{{\text{mf}}}} - \rho \alpha \mu b\sqrt \rho - GbD{\partial _x}\kappa  - \frac{\kappa }{\rho } GbA{\partial _x}\rho } \right\} \\
  \partial_t c & =  - {M_c}{\alpha _c}\Delta c + \beta {M_c}\Delta^2 \chi
\end{aligned}  }
\end{gather}
Ezek megoldását keressük egy egyensúlyi állapot körül,
\[\begin{aligned}
  \chi \left( {t,{\mathbf{r}}} \right) &  = {\chi _0} + \delta \chi \left( {t,{\mathbf{r}}} \right) \\ 
  \rho \left( {t,{\mathbf{t}}} \right) &  = {\rho _0} + \delta \rho \left( {t,{\mathbf{r}}} \right) \\ 
  \kappa \left( {t,{\mathbf{r}}} \right) &  = {\kappa _0} + \delta \kappa \left( {t,{\mathbf{r}}} \right) \\ 
  c\left( {t,{\mathbf{r}}} \right) &  = {c_0} + \delta c\left( {t,{\mathbf{r}}} \right).
\end{aligned} \]
Ha konstansok volnának, azok kielégítenék az egyenelteket, és a korábbi jelölések konveciója szerint 
\[\begin{aligned}
  {\partial _x}{\partial _y}\chi  = {\sigma _{xy}} = {\tau _0} \Rightarrow {\chi _0} &  = {\tau _0} \cdot xy \\ 
  \left\langle \rho  \right\rangle  &  = {\rho _0} \\ 
  \left\langle \kappa  \right\rangle  = 0 \Rightarrow {\kappa _0} &  = 0 \\ 
  \left\langle c \right\rangle  = {c_\infty } \Rightarrow {c_0} &  = {c_\infty }. \\ 
\end{aligned} \]
Az egyes variációk idő és térfüggését $\exp \left( {\frac{\lambda }{{{t_0}}} \cdot t + i\sqrt {{\rho _0}}  \cdot {\mathbf{kr}}} \right)$ alakban keresve a vizsgált alakok:
\begin{gather}
\tcbhighmath[borderline={1pt}{0pt}{black,solid}]{
\left( {\begin{array}{*{20}{c}}
  \chi  \\ 
  \rho  \\ 
  \kappa  \\ 
  c 
\end{array}} \right)\left( {t,{\mathbf{r}}} \right) = \left( {\begin{array}{*{20}{c}}
  {{\tau _0} \cdot xy} \\ 
  {{\rho _0}} \\ 
  0 \\ 
  {{c_\infty }} 
\end{array}} \right) + \left( {\begin{array}{*{20}{c}}
  {\delta {\chi _0}} \\ 
  {\delta {\rho _0}} \\ 
  {\delta {\kappa _0}} \\ 
  {\delta {c_0}} 
\end{array}} \right) \cdot \exp \left( {\frac{\lambda }{{{t_0}}} \cdot t + i\sqrt {{\rho _0}}  \cdot {\mathbf{kr}}} \right),}
\end{gather}
amelyben $1/{t_0} = {b^2}G{\rho _0}{M_0}$.

A behelyettesítés után a deriváló operátorok az alábbi sajátértéket veszik fel:
\[\begin{aligned}
  {\partial _t} &  \to \lambda /t_0 \\ 
  {\partial _y} &  \to i\sqrt {{\rho _0}}  \cdot {k_y} \\ 
  {\partial _y} &  \to i\sqrt {{\rho _0}}  \cdot {k_x} \\ 
  \Delta  &  \to {\left( {i\sqrt {{\rho _0}} } \right)^2} \cdot \left( {k_x^2 + k_y^2} \right) =  - {\rho _0} \cdot {k^2} \\ 
  {\Delta ^2} &  \to {\left( {i\sqrt {{\rho _0}} } \right)^4} \cdot \left( {k_x^4 + k_y^4 + 2k_x^2k_y^2} \right) = \rho _0^2 \cdot {k^4}. 
\end{aligned} \]

\subsubsection{Másodrendű feszültségtenzor}
\[\begin{aligned}
  0 &  =  - \frac{{1 - \nu }}{{2\mu }}{\Delta ^2}\left( {\delta \chi } \right) + b \cdot {\partial _y}\left( {\delta \kappa } \right) - \beta  \cdot \Delta \left( {\delta c} \right) \\ 
   &  =  - \frac{1}{{4\pi G}}\rho _0^2 \cdot {k^4}\left( {\delta \chi } \right) + b \cdot i\sqrt {{\rho _0}}  \cdot {k_y}\left( {\delta \kappa } \right) + \beta  \cdot {\rho _0} \cdot {k^2}\left( {\delta c} \right) \\ 
\end{aligned} \]
Ebből
\begin{equation} \label{eq:lin_ed_chi}
\tcbhighmath[borderline={1pt}{0pt}{blue,solid}]{
\delta \chi  = i\frac{{4\pi Gb}}{{\rho _0^{3/2}}}\frac{{{k_y}}}{{{k^4}}} \cdot \delta \kappa  + \frac{{4\pi G}}{{{\rho _0}}}\beta \frac{1}{{{k^2}}} \cdot \delta c.}
\end{equation}

\subsubsection{Diszlokációsűrűségek}
A teljes diszlokációsűrűség fejlődésére:
\[\begin{aligned}
  \frac{\lambda }{{{t_0}}}\delta \rho  &  =  - {M_0}b{\partial _x}\left\{ {\underbrace {\delta \kappa  \cdot {\partial _y}{\partial _y}\left( {xy \cdot {\tau _0} + \delta \chi } \right)}_{\delta \kappa  \cdot {\tau _0} + \delta \kappa  \cdot \delta \chi  \cdot  \ldots } - \underbrace {\frac{{\delta \kappa }}{\rho }GbD{\partial _x}\delta \kappa }_{ \propto \delta {\kappa ^2}} - GbA{\partial _x}\delta \rho } \right\} \\ 
   &  =  - {M_0}bi\sqrt {{\rho _0}}  \cdot {k_x}\left\{ {\delta \kappa  \cdot {\tau _0} - GbAi\sqrt {{\rho _0}}  \cdot {k_x}\delta \rho } \right\} \\ 
\end{aligned} \]
Ebből 
\begin{equation} \label{eq:lin_ed_rho}
\tcbhighmath[borderline={1pt}{0pt}{blue,solid}]{
0 = \left( {\lambda  + Ak_x^2} \right) \cdot \delta \rho  + i\frac{{{\tau _0}}}{{bG\sqrt {{\rho _0}} }}{k_x} \cdot \delta \kappa }
\end{equation}
Az előjeles diszlokációsűrűség fejlődésére pedig:
\[\begin{aligned}
  \frac{\lambda }{{{t_0}}}\delta\kappa  &  =  - {M_0}b{\partial _x}\left\{ \begin{gathered}
  \underbrace {\left( {{\rho _0} + \delta \rho } \right) \cdot {\partial _x}{\partial _y}\left( {xy \cdot {\tau _0} + \delta \chi } \right)}_{\underbrace {{\rho _0}{\tau _0}}_{{\text{konst}}{\text{.}}} + \delta \rho  \cdot {\tau _0} + {\rho _0}{\partial _x}{\partial _y}\chi  + \underbrace {\delta \rho  \cdot \delta \chi  \cdot  \ldots }_{{\text{magasabb rend}}}} - \left( {{\rho _0} + \delta \rho } \right)\alpha \mu b\underbrace {\sqrt {{\rho _0} + \delta \rho } }_{ \approx \left[ {1 + \delta \rho /\left( {2{\rho _0}} \right)} \right]\sqrt {{\rho _0}} } -  \hfill \\
  \quad \quad \quad \quad \quad \quad \quad \quad \quad \quad \quad \quad  - GbD{\partial _x}\delta\kappa  - \underbrace {\frac{{\delta \kappa }}{{{\rho _0} + \delta \rho }}GbA{\partial _x}\left( {{\rho _0} + \delta \rho } \right)}_{{\text{konst}}{\text{. deriv - ja}} + {\text{magasabb rend}}} \hfill \\ 
\end{gathered}  \right\} \\ 
   &  =  - {M_0}b{\partial _x}\left\{ {\delta \rho  \cdot {\tau _0} + {\rho _0}{\partial _x}{\partial _y}\delta \chi  - \left( {{\rho _0} + \delta \rho } \right)\alpha \mu b\left[ {1 + \delta \rho /\left( {2{\rho _0}} \right)} \right]\sqrt {{\rho _0}}  - GbD{\partial _x}\delta\kappa } \right\} \\ 
   &  =  - {M_0}bi\sqrt {{\rho _0}} {k_x}\left\{ {\delta \rho  \cdot {\tau _0} + {\rho _0}i\sqrt {{\rho _0}} {k_x}i\sqrt {{\rho _0}} {k_y}\delta \chi  - \frac{3}{2}\alpha \mu b\delta \rho \sqrt {{\rho _0}}  - GbDi\sqrt {{\rho _0}} {k_x}\delta\kappa } \right\} \\ 
\end{aligned} \]
Ebből
\begin{equation} \label{eq:lin_ed_kappa}
\tcbhighmath[borderline={1pt}{0pt}{blue,solid}]{
0 = \lambda \delta\kappa  + i\frac{{{\tau _0}}}{{bG\sqrt {{\rho _0}} }}{k_x} \cdot \delta \rho  - i\frac{{\rho _0^{3/2}}}{{bG}}k_x^2{k_y} \cdot \delta \chi  - i\frac{3}{2}\frac{{\alpha \mu b\sqrt {{\rho _0}} }}{{Gb\sqrt {{\rho _0}} }}{k_x} \cdot \delta \rho  + Dk_x^2 \cdot \delta \kappa.}
\end{equation}

\subsubsection{Oldott atom koncentráció}
\[\begin{aligned}
  \frac{\lambda }{{{t_0}}}\delta c &  =  - {M_c}{\alpha _c}\Delta \left( {\delta c} \right) + \beta {M_c}{\Delta ^2}\left( {xy \cdot {\tau _0} + \delta \chi } \right) \\ 
   &  =  - {M_c}{\alpha _c}\left( { - {\rho _0} \cdot {k^2}} \right)\delta c + \beta {M_c}\left( {\rho _0^2 \cdot {k^4}} \right) \cdot \delta \chi  \\ 
   &  = {M_c}{\alpha _c}{\rho _0}{k^2} \cdot \delta c + \beta {M_c}\rho _0^2{k^4} \cdot \delta \chi  \\ 
\end{aligned} \]
Ebből, új jelöléseket bevezetve:
\begin{equation} \label{eq:lin_ed_c}
\tcbhighmath[borderline={1pt}{0pt}{blue,solid}]{
0 = \left( {\lambda  - \frac{{{M_r}{\alpha _r}{k^2}}}{{{b^2}}}} \right) \cdot \delta c - \frac{{\beta {M_r}{\rho _0}}}{{{b^2}G}}{k^4} \cdot \delta \chi \qquad \frac{{{M_c}}}{{{M_0}}}: = {M_r}\quad \frac{{{\alpha _c}}}{G}: = {\alpha _r}.}
\end{equation}

\subsubsection{Másodrendű feszültségtenzor eliminálása}
\Cref{eq:lin_ed_kappa,eq:lin_ed_c} tartalmaznak $\delta\chi$ függést, amit \cref{eq:lin_ed_chi} segítségével eliminálhatunk. A kapott egyenletek a ${\tau _r} = {\tau _0} - {\tau _f}$ jelöléssel, \cref{eq:lin_ed_rho}-vel együtt:
\begin{gather}\begin{aligned}
  0 &  = \left( {\lambda  + Ak_x^2} \right) \cdot \delta \rho  + i\frac{{{\tau _0}}}{{bG\sqrt {{\rho _0}} }}{k_x} \cdot \delta \kappa  \\ 
  0 &  = i\left( {{\tau _r} - \frac{{{\tau _f}}}{2}} \right)\frac{1}{{bG\sqrt {{\rho _0}} }}{k_x} \cdot \delta \rho  + \left( {\lambda  + Dk_x^2 + 4\pi \frac{{k_x^2 k_y^2}}{{{k^4}}}} \right)\delta \kappa  - i4\pi \beta \frac{{\sqrt {{\rho _0}} }}{b}\frac{{k_x^2{k_y}}}{{{k^2}}} \cdot \delta c \\ 
  0 &  =  - i4\pi \frac{{\beta {M_r}}}{{b\sqrt {{\rho _0}} }}{k_y} \cdot \delta \kappa  + \left[ {\lambda  - \left( {{\alpha _r} + 4\pi {\beta ^2}} \right)\frac{{{M_r}{k^2}}}{{{b^2}}}} \right] \cdot \delta c. \\ 
\end{aligned}
\end{gather}
Mátrixos írásmódban kapjuk, hogy 
\begin{equation}
\tcbhighmath[borderline={1pt}{0pt}{black,solid}]{
\left({\begin{array}{*{20}{c}}
  {\lambda  + Ak_x^2}&{i\frac{{{\tau _0}}}{{bG\sqrt {{\rho _0}} }}{k_x}}&0 \\ 
  {i\left( {{\tau _r} - \frac{{{\tau _f}}}{2}} \right)\frac{1}{{bG\sqrt {{\rho _0}} }}{k_x}}&{\lambda  + Dk_x^2 + 4\pi \frac{{k_x^2 k_y^2}}{{{k^4}}}}&{ - i4\pi \beta \frac{{\sqrt {{\rho _0}} }}{b}\frac{{k_x^2{k_y}}}{{{k^2}}}} \\ 
  0&{ - i4\pi \frac{{\beta {M_r}}}{{b\sqrt {{\rho _0}} }}{k_y} }&{\lambda  - \left( {{\alpha _r} + 4\pi {\beta ^2}} \right)\frac{{{M_r}{k^2}}}{{{b^2}}}} 
\end{array}} \right)\left( {\begin{array}{*{20}{c}}
  {\delta \rho } \\ 
  {\delta \kappa } \\ 
  {{\delta _c}} 
\end{array}} \right) = 0.}
\end{equation}
\end{document}